% \appendix
\renewcommand{\thechapter}{A}
\chapter{Usage of FIPS 140-2 functionality of sec-certs}

\definecolor{codegreen}{rgb}{0,0.6,0}
\definecolor{codegray}{rgb}{0.5,0.5,0.5}
\definecolor{codepurple}{rgb}{0.58,0,0.82}
\definecolor{backcolour}{rgb}{0.95,0.95,0.92}

\lstdefinestyle{mystyle}{
    backgroundcolor=\color{white},   
    commentstyle=\color{codegreen},
    keywordstyle=\color{magenta},
    numberstyle=\tiny\color{codegray},
    stringstyle=\color{codepurple},
    basicstyle=\ttfamily\footnotesize,
    breakatwhitespace=false,         
    breaklines=true,                 
    captionpos=b,                    
    keepspaces=true,                          
    showspaces=false,                
    showstringspaces=false,
    showtabs=false,                  
    tabsize=2
}

\lstset{style=mystyle}

% \section*{Usage of FIPS 140-2 functionality of sec-certs}

The Python module for analysis of FIPS 140-2 certificates, described in the thesis, is integrated in the \texttt{sec-certs} tool~\cite{sec-certs}. Current version of \texttt{sec-certs} acts like a Python library to be used in users' own scripts. 

\section{Installation}

The tool requires \texttt{Python>=3.8} and accessible \texttt{pdftotext} binary. The tool can be set up in a Python virtual environment and \texttt{setup.py}:

\begin{lstlisting}[language=bash]
    $ python3 -m venv venv
    $ source venv/bin/activate
    $ pip install -e .
\end{lstlisting}
This installs all neccessary python libraries needed for \texttt{sec-certs}.

\section{Folder structure}

\texttt{sec-certs} files related to this thesis are the following:
\begin{itemize}
    \item \texttt{sec\_certs} -- folder containing the Python module files.
    \begin{itemize}
        \item \texttt{dataset} -- folder containing the classes used for dataset manipulation
            \begin{itemize}
                \item \texttt{fips.py} -- file containing the \texttt{FIPSDataset} class, used to handle the main dataset.
                \item \texttt{fips\_algorithm.py} -- file containing the \texttt{FIPSAlgorithmDataset} class, used to handle the algorithm implementations dataset.
            \end{itemize}
            Both these classes inherit from the \texttt{Dataset} class in \texttt{dataset.py}.
        \item \texttt{certificate} -- folder containing the classes used for manipulation with certificate objects used in datasets
            \begin{itemize}
                \item \texttt{fips.py} -- file containing the \texttt{FIPSCertificate} class and \texttt{FIPSAlgorithm} as a subclass.
            \end{itemize}
            The \texttt{FIPSCertificate} class inherits from the \texttt{Certificate} class in \texttt{certificate.py}.
        \item \texttt{configuration.py} -- class containing the global config loaded from the \texttt{YAML} file.
        \item \texttt{cert\_rules.py} -- file containing the regular expressions used during the keyword extraction. The variables including \texttt{fips} in their name are used in the functionality described in this thesis.
        \item \texttt{helpers.py} -- some of the helper functions used during the data extraction.
    \end{itemize}
    \item \texttt{examples} -- folder containing the provided entrypoints for \texttt{sec-certs}. Explained in \myref{Section}{entrypoint}.
    \item \texttt{tests} -- folder containing tests to \texttt{sec-certs}. The files related to the FIPS 140-2 functionality contain \texttt{fips} in their name.
\end{itemize}

\section*{Basic usage}

\texttt{sec-certs} is supposed to be used as a python module. We provided examples on the using of the CC and FIPS functionality in the \texttt{examples/} folder. The \texttt{fips\_oop\_demo.py} script provides a CLI and a main entrypoint for the FIPS 140-2 functionality of \texttt{sec-certs}. Further details of FIPS functionality are explained in \myref{Listing}{code}.

\begin{lstlisting}[language=python,caption=Example of using FIPS 140-2 functionality of \texttt{sec-certs}, label={code}]
    """
    Load config YAML file and make it globally available in the 'config' variable. 
    The example config file is available in 'sec-certs/settings.yaml' file. 
    All presented values are required.
    :param filepath     path to config YAML file
    """
    config.load(filepath="path/to/config/file")
    """
    Create empty dataset object. 
    The 'FIPSDataset' class is implemented in 'sec-certs/dataset/fips.py'.
    :param certs        Dictionary containing <certificate_id, 'FIPSCertificate'> pairs.
    :param root_dir     Path to the directory where all the metadata will be saved. 
    :param name         Name of the dataset.
    :param description  Description of the dataset.
    """
    dataset = FIPSDataset(certs={}, root_dir="path/to/root/dataset/directory", 
                        name='dataset_name', description='dataset description')
    """
    Extracts the information from CMVP web pages and populates the dataset.
    This method provides the 'web_scan' functionality:
        -   downloads the list of CMVP validated modules from the search pages
        -   creates dataset entries
        -   downloads HTML page and security policy document for each available module
        -   extracts information from HTML pages
    :param redo         Extract the information from webpages from scratch.
    :param json_file    Optional - path to saved json file created during previous calculations. 
                        If provided, the dataset is loaded from this file and is only updated.
                        If not provided, the dataset is created all over again.
                        Default value is "dataset.root_dir / 'fips_full_dataset.json'"
    :param test         Optional - provides path to modified CMVP search page, used only in tests to check whether the functionality is correct for a subset of all certificates.
    :param update_json  Optional - if true, the dataset in "dataset.root_dir / 'fips_full_dataset.json'" is updated at the end of this method.
                        If False or not provided, the saved dataset is not updated.                    
    """
    dataset.get_certs_from_web(redo, json_file, test, update_json)
    """
    This method uses pdftotext to convert the security policy PDFs to text files.
    :param update_json  Same functionality as in 'get_certs_from_web()' method.
    """
    dataset.convert_all_pdfs(update_json)
    """
    Extracts the information from converted security policy documents using regular 
    expressions.
    This method provides the 'pdf_scan' functionality:
        -   extracts non-specific keywords used in further analyses
        -   extracts FIPS 140-2 specific keywords used to find references
        -   saves the modified files in dataset.root_dir / 'fragments/' directory
    The extraction is done using multiprocessing. 
    :param redo         Extract the keywords from converted security policies from scratch.
    :param update_json  Same functionality as in 'get_certs_from_web()' method.
    """
    dataset.extract_keywords(redo, update_json)
    """
    This method is used to extract algorithm implementation IDs from tables in 
    security policy PDFs.
    The process may fail for some security policies, so a list of paths
    to not parsed files is returned.
    :param high_precision   This parameter specifies, whether to use a heuristic to find the pages with tables in the PDF, scanning only them. However, this is often not extensive and the results are not precise for some certificates.
                            If 'True', the scan is performed on all pages of every PDF.
                            If 'False', heuristics are used to find the pages containing tables.
    :param update_json      Same functionality as in 'get_certs_from_web()' method.
    :return                 List of paths of security policy PDFs that were not parsed.
    """
    not_decoded_files = dataset.extract_certs_from_tables(high_precision, update_json)
    """
    The algorithm information can be reused from previous runs and can be loaded in 
    the 'get_certs_from_web()' method. However, to always work with the most recent 
    information, it is recommended to always update the algorithm implementation dataset.
    """
    aset = FIPSAlgorithmDataset({}, Path(dataset.root_dir / 'web/algorithms'), 'algorithms', 'sample algs')
    aset.get_certs_from_web()
    dataset.algorithms = aset
    """
    This method is used to combine results from 'web_scan' and 'pdf_scan'.
    This method is redone everytime, even though the results may be saved.
    It implements the 'processed' functionality:
        -   sanitizes and unifies the extracted algorithm implementation and module IDs
        -   merges the results of 'web_scan' and 'pdf_scan'
        -   removes false positives by using heuristics
        -   finds the references for each certificate
    The search for not found algorithms from web in the PDF is also ran here.
    :param update_json      Same functionality as in 'get_certs_from_web()' method.
    """
    dataset.finalize_results(update_json)
    """
    This method is used to plot graphviz graphs of dependencies between the certificates.
    Three graphs are plotted - one only from 'web_scan', one only from 'pdf_scan'
    and one from processed results.
    The graphs are saved in 'dataset.root_dir / <graph_type>.png', where <graph_type>
    is one of ['full_graph', 'web_only_graph', 'pdf_only_graph'].
    :param show     Display the graphs right after plotting.
    """
    dataset.plot_graphs(show)
\end{lstlisting}

\section{Entrypoint}\label{entrypoint}

As we mentioned earlier, we provided \texttt{fips\_oop\_demo.py} as an entrypoint for the FIPS 140-2 functionality. The script provides CLI and logging for the functionality described in \myref{Listing}{code}.

The script accepts following arguments, all of which are optional:
\begin{itemize}
    \item \texttt{-{}-config-file} -- specifies the path to the YAML config file
    \item \texttt{-{}-json-file} -- specifies the path to the saved dataset from previous runs
    \item \texttt{-{}-no-download-algs} -- use the cached algorithm dataset, do not update it from web
    \item \texttt{-{}-redo-web-scan} -- flag, specifies whether to redo \texttt{web\_scan} information extraction from scratch
    \item \texttt{-{}-redo-keyword-scan} -- flag, specifies whether to redo \texttt{pdf\_scan} information extraction from scratch
    \item \texttt{-{}-higher-precision-results} -- flag, specifies whether to search all pages in PDFs for tables in order to obtain more precise results 
\end{itemize}
It is recommended to run the script without arguments the first time it is run.

The full extraction and analysis process presented in \texttt{fips\_oop\_demo.py} takes four to six hours and about four gigabytes of disk space is occupied by downloaded metadata used in computations.

More information on the \texttt{fips\_oop\_demo.py} entrypoint is available in the main README of \texttt{sec-certs}.

\renewcommand{\thechapter}{b}

\chapter{Data attachments}
\begin{itemize}
    \item \texttt{sec-certs} -- a snapshot of the \texttt{sec-certs} tool from the \texttt{sec-certs} github repository~\cite{sec-certs}.
    \item \texttt{graphs} -- folder containing the dot graphs of CMVP cryptographic module references plotted using the \texttt{sec-certs} tool.
\end{itemize}