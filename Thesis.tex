\documentclass{mimosis}

\usepackage{metalogo}
\usepackage{xargs} % Use more than one optional parameter in a new commands
\usepackage{indentfirst}

\usepackage[colorinlistoftodos,prependcaption,textsize=tiny]{todonotes}
\newcommandx{\unsure}[2][1=]{\todo[linecolor=red,backgroundcolor=red!25,bordercolor=red,#1]{#2}}
\newcommandx{\change}[2][1=]{\todo[linecolor=blue,backgroundcolor=blue!25,bordercolor=blue,#1]{#2}}

\newcommandx{\info}[2][1=]{\todo[linecolor=OliveGreen,backgroundcolor=OliveGreen!25,bordercolor=OliveGreen,#1]{#2}}
\newcommandx{\improvement}[2][1=]{\todo[linecolor=Plum,backgroundcolor=Plum!25,bordercolor=Plum,#1]{#2}}
\newcommandx{\thiswillnotshow}[2][1=]{\todo[disable,#2]{#1}}
%

\usepackage{float}
\usepackage{geometry}
\usepackage{babel}
\usepackage{booktabs}
\newcommand{\ra}[1]{\renewcommand{\arraystretch}{#1}}

\graphicspath{ {./img/} }

\newcommand{\alphabet}{%
  abcdefghijklmnopqrstuvwxyz%
}
\newlength{\textW}
\setlength{\textW}{\widthof{\alphabet}* \real{2.7}} % 2.5
\geometry{textwidth=\textW,}
\newcommand{\myref}[2]{\hyperref[#2]{#1~\ref*{#2}}}

%%%%%%%%%%%%%%%%%%%%%%%%%%%%%%%%%%%%%%%%%%%%%%%%%%%%%%%%%%%%%%%%%%%%%%%%
% Some of my favorite personal adjustments
%%%%%%%%%%%%%%%%%%%%%%%%%%%%%%%%%%%%%%%%%%%%%%%%%%%%%%%%%%%%%%%%%%%%%%%%
%
% These are the adjustments that I consider necessary for typesetting
% a nice thesis. However, they are *not* included in the template, as
% I do not want to force you to use them.

% This ensures that I am able to typeset bold font in table while still aligning the numbers
% correctly.
\usepackage{etoolbox}
\usepackage{listings}

\usepackage{siunitx}

\sisetup{%
  detect-all           = true,
  detect-family        = true,
  detect-mode          = true,
  detect-shape         = true,
  detect-weight        = true,
  % detect-inline-weight = math,
}

%%%%%%%%%%%%%%%%%%%%%%%%%%%%%%%%%%%%%%%%%%%%%%%%%%%%%%%%%%%%%%%%%%%%%%%%
% Hyperlinks & bookmarks
%%%%%%%%%%%%%%%%%%%%%%%%%%%%%%%%%%%%%%%%%%%%%%%%%%%%%%%%%%%%%%%%%%%%%%%%

\usepackage[%
  colorlinks = true,
  citecolor  = RoyalBlue,
  linkcolor  = RoyalBlue,
  urlcolor   = RoyalBlue,
  unicode,
  ]{hyperref}


\newcommand{\smallquote}[1]{
    \begin{center}
        \begin{minipage}{0.5\textwidth}
            \begin{small}
                #1
            \end{small}
        \end{minipage}
        \vspace{0.5cm}
    \end{center}
}

\usepackage{bookmark}

\usepackage[all]{nowidow}

%%%%%%%%%%%%%%%%%%%%%%%%%%%%%%%%%%%%%%%%%%%%%%%%%%%%%%%%%%%%%%%%%%%%%%%%
% Bibliography
%%%%%%%%%%%%%%%%%%%%%%%%%%%%%%%%%%%%%%%%%%%%%%%%%%%%%%%%%%%%%%%%%%%%%%%%
%
% I like the bibliography to be extremely plain, showing only a numeric
% identifier and citing everything in simple brackets. The first names,
% if present, will be initialized. DOIs and URLs will be preserved.

\usepackage[%
  autocite     = plain,
  backend      = biber,
  doi          = true,
  url          = true,
  giveninits   = true,
  hyperref     = true,
  maxbibnames  = 99,
  maxcitenames = 99,
  sortcites    = true,
  style        = iso-numeric,
  ]{biblatex}
\usepackage{xurl}

\input{bibliography-mimosis}
\addbibresource{Thesis.bib}

%%%%%%%%%%%%%%%%%%%%%%%%%%%%%%%%%%%%%%%%%%%%%%%%%%%%%%%%%%%%%%%%%%%%%%%%
% Fonts
%%%%%%%%%%%%%%%%%%%%%%%%%%%%%%%%%%%%%%%%%%%%%%%%%%%%%%%%%%%%%%%%%%%%%%%%

\ifxetexorluatex
  % \setmainfont{Minion Pro}
  %  \setmainfont{Baskerville}
    \setmainfont{Palatino}
    \setsansfont{Palatino}
    \setmonofont{Source Code Pro}
\else
  \usepackage[lf]{ebgaramond}
  \usepackage[oldstyle,scale=0.75]{sourcecodepro}
  \singlespacing
\fi

\renewcommand{\th}{\textsuperscript{\textup{th}}\xspace}

\newacronym{AI}{AI}{Artificial Intellingence}
\newacronym{CAM}{CAM}{Class Activation Map}
\newacronym{DL}{DL}{Deep Learning}
\newacronym{ML}{ML}{Machine Learning}
\newacronym{GAP}{GAP}{Global Average Pooling}
\newacronym{GMP}{GMP}{Global Max Pooling}
\newacronym{LRP}{LRP}{Layer-Wise Relevance Propagation}
\newacronym{XAI}{XAI}{Explainable Artificial Intelligence}
\newacronym{CNN}{CNN}{Convolutional Neural Network}
\newacronym{NN}{NN}{Neural Network}
\newacronym{WSI}{WSI}{Whole Slide Image}
\newacronym{FC}{FC}{Fully Connected}
\newacronym{GPU}{GPU}{Graphics Processing Unit}
\newacronym{RAM}{RAM}{Random-Access Memory}
\newacronym{ROAR}{ROAR}{Remove and Retrain}
\newacronym{ROAD}{ROAD}{Remove and Debias}


\usepackage{textcomp}
\usepackage{tikz}
\usepackage[framemethod=TikZ]{mdframed} % Use TikZ for framing
\usepackage{xcolor}

\usepackage{caption}
\captionsetup{format=plain, width=0.85\textwidth, labelfont=bf}

\usepackage{svg}

\newmdenv[
  topline=false,
  bottomline=false,
  skipabove=\topsep,
  skipbelow=\topsep,
  innerleftmargin=10pt,
  innerrightmargin=10pt,
  leftline=true, % Enable the left line
  rightline=true, % Enable the right line
  linewidth=2pt, % Thickness of the lines
  linecolor=RoyalBlue, % Color of the lines
]{doublebar}


\makeindex
\makeglossaries

\setlength{\parindent}{0pt} % Removes the paragraph indentation
\setlength{\parskip}{6pt} % Replace 10pt with the desired space

\DeclareMathOperator{\relu}{ReLU}

\hyphenation{RationAI}
\hyphenation{pseu-do-nym-ized}

%%%%%%%%%%%%%%%%%%%%%%%%%%%%%%%%%%%%%%%%%%%%%%%%%%%%%%%%%%%%%%%%%%%%%%%%
% Incipit
%%%%%%%%%%%%%%%%%%%%%%%%%%%%%%%%%%%%%%%%%%%%%%%%%%%%%%%%%%%%%%%%%%%%%%%%

\title{Interpretation Techniques for Deep Neural Networks in Digital Histopathology}
\author{Martin Krebs}

% This ensures that the subsequent sections are being included as root
% items in the bookmark structure of your PDF reader.
\begin{document}
\frontmatter
    \include{Sources/Title}
    \chapter*{Declaration}

\noindent
Hereby I declare, that this paper is my original authorial work, which I
have worked out on my own. All sources, references and literature used or
excerpted during elaboration of this work are properly cited and listed in
complete reference to the due source.

\vspace{1cm}
\begin{flushright}
    Martin Krebs
\end{flushright}
\vfill
\textbf{Advisor:} RNDr. Vít Musil, Ph.D.
\newline
\textbf{Consultant:} doc. RNDr. Tomáš Brázdil, Ph.D.

    \chapter*{Acknowledgements}
\noindent
\todo{Write acknowledgements}
    \chapter*{Abstract}

Our goal is to generate explanations that help understand prostate cancer presence predictions in a reasonable time.
We review several popular explainability methods that produce visually similar results to the current, notably slow solution based on Occlusion.
To ensure the suitability of our candidate methods, we carefully establish exhaustive quantitative and qualitative benchmarks.
Our evaluated methods achieve $2$ orders of magnitude faster computation time and $50$ times better GPU utilization without sacrificing faithfulness or localization capabilities.
Furthermore, a domain expert considers two methods as viable alternatives.
With all auxiliary pipeline processing, the methods selected as suitable allow us to generate explanations for the used dataset in about $2$ hours, compared to the $3.3$ days needed using Occlusion.

\section*{Keywords}
digital histopathology, convolutional neural networks, explainable artificial intelligence, occlusion, CAM, HiResCAM, performance, faithfulness, localization

    
    % \tableofcontents
    {\small\tableofcontents}
    % \listoffigures
    \glsaddall
    \printglossary[type=\acronymtype]

    \chapter{Introduction}\label{chap:introduction}

With deep learning and neural networks in the spotlight of contemporary research, we see their applications in a wide range of areas.

However, in places where the capabilities of deep learning models could be the most helpful, deployed neural networks suffer from their ``black box'' tag and the lack of insight behind their decisions.

In such critical areas as medicine, self-driving, or law, it is essential to have an accurate proxy to shed light on the model's decision process.

This thesis aims to find methods suitable for such reasoning.
Occlusion, a faithful method tested by Gallo et al. in \cite{gallo}, is too slow to be used in real time.
Thus, we begin our search for a faster method that does not compromise the usability and faithfulness of the explanations produced.

To find such a method, we need to understand our domain and its limitations.
Therefore, we start with an overview of essential concepts and nomenclature.
We then motivate the need for such an explaining proxy and review several well-established methods, carefully selected based on previous work of the RationAI group in  \cite{gallo, bajger-grad-cam, krajnansky-grad-cam, hruska-grad-cam}.
In order to verify that the outlined methods are indeed suitable, we establish and evaluate them against tailored quantitative and qualitative benchmarks.



\mainmatter
  \chapter{Deep Learning in Digital Histopathology}

We begin by introducing the field of histopathology and its concerns, alongside with tools and methods used in contemporary clinical practice.
We show how digitization helped to reduce the logistical complexity of histopathologists workflow, and how decision support systems can further reduce the required human labor.
We follow by introducing feed-forward artificial neural networks, with focus on convolutional networks, which are exceptional at tackling various computer vision problems.

\section{Histopathology}

Histopathology is a discipline concerned with study of diseases of tissue.
This involves, but is not limited to cancer detection and prediction [], infectious or inflammatory disease diagnostics [] and study of brain-degenerative diseases such as Parkinson's or Alzheimer's [].

Histopathologists are medically qualified physicians, who inspect tissue taken from patients.
Their expertise is essential in identifying cellular and tissue anomalies that could indicate a range of medical conditions.
Histopathologists often cooperate with other doctors, providing insights to help set the direction of further patients care [].

\section{Temporal and spatial limitations of traditional Histopathology}

Traditionally, to get tissue from patient to histopathologist, tedious logistic process involving several people is necessary.
Surgeon needs to extract tissue samples from patient.
Extracted tissue is then sent to specialized laboratory for processing -- in the laboratory, tissue samples are infused with mix of chemicals and embedded into paraffin wax block.
The parafin block is then thinly sliced into sections of approx. $3$ microns, and those section are laminated onto a glass slide.
Glass slides are further stained with hematoxilin/eosin (or other compound) [] to enhance contrast between different cellular structures.
After the lab processing, slides are delivered to histopathologist for a review.

While we currently cannot replace surgeon performing the biopsy or lab workers staining and embedding the tissue, we can address logistic challenges of moving glass slides to histopathologist.
Having a physical slide suffers from several inefficiencies. 
A slide can be studied only by one histopathologist at a time and if a second opinion from a different histopathologist is required, the glass slide must be conveyed to the respective clinic.
This throttles the diagnosis process and leads to longer waiting times for a patient.

\section{Digital histopathology}
% https://pathsocjournals.onlinelibrary.wiley.com/doi/10.1002/path.5388

Digital histopathology aims to reduce the logistic overhead caused by physical copies of glass slides.
After the tissue is extracted and prepared, instead of shipping it to a histopathologist, it is scanned using specialized lenses resulting in a high-resolution digital image, called \emph{Whole Slide Image} (WSI) [].

This image is then uploaded to an aggregator server, which enables real-time sharing of slides and parallel cooperation of multiple clinicians.
Histopathologists then inspect WSI in a dedicated browsers on their computer monitors, instead of looking at the glass slide under a microscope. Figure \ref{fig:xopat} shows WSI look in a dedicated browser developed by RationAI group.

Even though contemporary digital pathology systems provide a significant speed up in pathologists day to day work, we can further optimize another productivity metric – pathologists time spent on inspection of WSIs.
Recently, researches and various companies made attempts to employ machine learning systems to further aid pathologist during the diagnosis process. 

\begin{figure}
    \begin{center}
    \begin{minipage}{0.75\textwidth}
      \includegraphics[width=\textwidth]{img/xopat.png}
    \end{minipage}
    \caption{xOpat [] WSI browser with a sample of prostate tissue. Histopathologist is able to move and zoom the tissue, allowing him to quickly navigate in the WSI. Having digitized copies allows for layering of arbitrary annotations on top of the WSI, providing great collaborative capabilities.}
    \label{fig:xopat}
    \end{center}
\end{figure}
\todo{Maybe add border?}

\subsection*{Decision Support System}

A computer system, which aids human to make a decision while performing a particular task is referred to as decision support system [].
These systems are utilized across a wide array of applications, including high-stake environments such as investment banking [], autonomous driving [] or military and defense []. In digital histopathology, such systems are usually utilized to help with operational processes [].
With deep learning in the spotlight of today's research, new possibilities emerge.

New systems could be used to enhance tissue diagnostic process by assisted diagnosis, or used to discover previously unrecognized features in large sets of data, incomprehensible by a single expert []. %% https://pathsocjournals.onlinelibrary.wiley.com/doi/10.1002/path.5388
Random forests [], support vector machines [] and various neural network architectures [] are all attempts of such utilization.
Those systems often provide real-time results and human-like performance, demonstrating that further research in this area can bring significant improvement to the contemporary processes [].

%% decision trees - https://www.researchgate.net/publication/363350224_Random_forest_modelling_demonstrates_microglial_and_protein_misfolding_features_to_be_key_phenotypic_markers_in_C9orf72-ALS
%% svm - https://www.ncbi.nlm.nih.gov/pmc/articles/PMC1924513/pdf/1471-2121-8-S1-S8.pdf
%% neural networks - https://arxiv.org/pdf/2312.02225.pdf
%%                 - https://www.sciencedirect.com/science/article/pii/S2666827021000992

\section{Deep Learning}
%% Brief intro to what is deep learning, methods and techniques? Common use cases of deep learning in digital histopathology.

Area of machine learning encapsulating neural networks (NNs) is referred to as deep learning (DL).
Methods and algorithms employed by DL achieve remarkable results across various domains, including computer vision [imagenet?], games [alphago], weather forecast [] and natural language processing [gpt]. 
Introduced in 1943 by McCulloch and Pitts [] with goal of creating a computational unit resembling a neuron in human brain, neural networks have come a long way to the prominent place they occupy today.

%% forecast - https://www.nature.com/articles/s41586-023-06185-3

\subsection*{Feed-forward Neural Network}

Feed-forward networks are considered the foundation of contemporary models.
They get their name from one-directional flow of information inside the network.
We can see feed-forward networks as a function $f$ that maps real-valued input vector $x$ from input space to a value $y$ from output space.
Throughout this thesis, we will only consider this type of networks.

A common approach to examine neural network architectures is to see them a sequence of layers.
Feed-forward neural network is a sequences of layers, and each layer $l$ resembles an intermediate function $f^l$.
Suppose our network computes a function $f$ and that it is composed of $L$ layers.
For any given input vector $x$, the computation performed by the network can be expressed as a composition of these intermediate layer functions, yielding
\begin{equation}
    f(x) = f^L(f^{L-1}(\cdots f^1(x))).
\end{equation}
In this notation, layer $1$ serves as the input layer, which directly receives the input vector. The final or $L$-th layer is known as the output layer, which outputs the network's prediction or decision. All in-between layers are referred to as hidden layers and they transform one internal representation to another, per [].

Even though ... shows that one hidden layer is all you need, neural networks typically employ multiple hidden layers.
The motivation is that each layer can be seen, as if it captures certain abstractions or representations from its input.
Deeper layers are therefore able to build more complex representations, by utilizing abstractions captured by previous layers.

Multi-Layer Perceptrons (MLPs) are a foundational architecture in the development of feed-forward neural networks.
In a MLP, the basic building block is a neuron.
Neurons are arranged into layers to form the final network.
They receive input signals, process them, and produce output signals that are passed to subsequent neurons in the following layer.
The processing consists of computing inner potential $\xi$, which is a weighted average of neuron's inputs.
Inner potential is then passed to the activation function \footnote{Heaviside step function was the first activation function to be used. This led to a number of problems [] and because of their importance, activation functions are vital part of research interest up to this day [].}, commonly denoted as $\sigma$. 
Given a neuron $i$ with activation function $\sigma_i$, if the neuron expects $K$ input features, the output $y_i$ is expressed as
\begin{equation}
    y_i = \sigma_i(\sum_{k=1}^K w_{ik}x_k + b_i)
\end{equation}
where weight $w_ik$ connects $k$-th input feature to the neuron $i$ and $b_i$ is constant bias term, improving neuron's modeling capabilities.
If all neurons in a given layer have a incoming weights from all neurons in the previous layer, we say that $l$ is fully-connected (FC).

Weights and biases are called trainable parameters, denoted as $\theta$.
Deep learning aims to make parameters useful.
Since a network computes a function $f$, with specific training, we can make network learn to approximate any desired function $f^*$ within a certain tolerance [all you need].
To train a network, we leverage large amount of data to adjust trainable parameters to minimize difference between $f$ and $f^*$.
The difference is captured using loss function $\ell$, and the process of minimizing $\ell$ is typically performed iteratively using backpropagation and training algorithm such as stochastic gradient descent.
More on neural networks training can be found in [].

Despite deep dive to training is out of scope of this thesis, we need to familiarize ourselves with the notion of partial derivatives.
Partial derivatives are commonly used to find a direction against which we move certain weight $w$ to minimize the $\ell$, if all other parameters remain fixed.
We denote the partial derivative of $\ell$ with respect to $w$ as $\frac{\partial \ell}{\partial w}$.
Building on the concept of partial derivatives, we can compute the gradient.
In the context of a deep learning model's loss function, the gradient $\nabla_{\theta}\ell$ is a vector of partial derivatives of $\ell$ with respect to all the parameters within the model
\begin{equation}
    \nabla_{\theta}\ell = \left( \frac{\partial \ell}{\partial w_1}, \frac{\partial \ell}{\partial w_2}, \cdots, \frac{\partial \ell}{\partial w_{|\theta|}} \right).
\end{equation}
This gradient points in the direction of the steepest ascent in the loss function's value.
Thus, to minimize, we update the parameters in the opposite direction of the gradient.
As we will see in Chapter 3, partial derivatives can also be leveraged to compute importance of features or networks parameters.
\todo{ref}
%% goodfellow kniha

\begin{figure}
    \begin{center}
    \begin{minipage}{.75\textwidth}
      \includegraphics[width=\textwidth]{img/nn.png}
    \end{minipage}
    \caption{Architecture of a MLP with one hidden layer. Circles represent layer inputs. Edges represent layer weights. Biases are omitted for simplicity. Notice, that each neuron in hidden and output layer share a weight with all neurons in previous layer -- therefore are fully-connected.}
    \label{fig:simple-mlp}
    \end{center}
\end{figure}

Even though MLPs demostrate impressive results on tasks previously deemed impossible for computers, they come with certain setbacks.
If we only use FC layers, even small contemporary architectures such as ImageNET would have unimaginable number of trainable parameters.
This led to development of new architectures, tailored to specific domain needs.
Despite shift from using FC layers only, MLP stood its ground and to this day and is an essential part of various state of the art neural network models [gpt, alexnet].

\subsection*{Convolutional Neural Network}
%% Architecture
Architecture introduced specifically to solve various computer vision problems adds two additional layer types --- convolutional and pooling.
These layers help to capture patterns in input features, as well as reduce size of the network []. 
%% - https://arxiv.org/pdf/1511.08458.pdf

%% TODO: Why it suits them for image processing

\subsubsection{Convolutional Layer}

The convolutional layer searches for visual patterns in its input features.
This type of layer utilizes trainable filters (sometimes called kernels) to detect the patterns.
The filter is typically smaller in dimensions than input, and instead of interacting with the whole input at once using weights, as is the case for fully-connected layers, the filter with its own weights is systematically slid across the input.
Weights of a filter are convolved with the corresponding input data segments, yielding an \emph{activation map}.
Convolution is a linear operation, which chains element-wise product of the filter and receptive field, followed by summing the product into one value. Given a filter $F$ and receptive field $I$, convolutions is defined as
\todo{wights to neurons as in FC layer, convolved, produce activation and feed to the next layer.}
\begin{equation}
    F * I = F_1 I_1 + \cdots + F_n I_n
\end{equation}
where $F_k$ and $I_k$ are individual spatial scalar features in filter and receptive field, respectively. This process in visually depicted in Figure ...

\todo{convolution from here file:///Users/krebso/Downloads/s13244-018-0639-9.pdf}
\todo{Filters are smaller than input}
Resulting activation map can be seen as a evidence for presence of a shape, detected by the filter in the input data. Sliding filter through features gives us spatial invariance -- no matter where in the input the pattern is, it will get detected and reflected in the respective activation map, something very hard to achieve using fully-connected layer.
Therefore it is commonly said, that the filter has "shared weights".

When incorporating a convolutional layer into a neural network, instead of specifying the number of neurons as with FC layers, the crucial parameter to define is the filter shape.
This shape determines the receptive field's size -- how much of the input the filter can see at any given time.
In addition we usually set two other parameters: \emph{padding} and \emph{stride}.
Padding is a value we add as a border around the input features, allowing the filter to cover the edges and thus reducing information loss.
Stride controls how much we shift the filter after each convolution.
For visual representation of these concepts see Figure \ref{fig:cnn-convolution}.


\begin{figure}
    \begin{center}
    \begin{minipage}{0.75\textwidth}
      \includegraphics[width=\textwidth]{img/cnn-conv.png}
    \end{minipage}
    \caption{Example of simple calculation within convolution layer. Filter detects diagonal edge of lenght 3 pixels. Stride and padding are both set to 1 pixel and zero is used as the padded value. The result is passed through ReLU activation function.}
    \label{fig:cnn-convolution}
    \end{center}
\end{figure}

Convolutional layer consists of multiple units.
Each unit has its own filter and produces unique activation map.
The idea is that during training, each filter learns to recognize a different pattern.
This gives a single layer capabilities to detect multiple patterns and similarly to its fully-connected counterpart -- it allows subsequent layers of the network to build upon captured abstractions and ultimately "understand" and represent high level concepts.
Given a layer with $n$ units, we will denote $k$-th filter as $F^k$ and $k$-th activation map as $A^k$.
Individual activation in spatial position $x, y$ will be denoted as $A^k_{xy}$.

It is important to note, that sharing weights has not only effects of identifying patterns regardless of their position in the input -- it reduces the size of networks as well.
Given $512$ input and output features, a fully connected layer needs $262,144$ weights to propagate the information.
On the other hand, convolutional layer with $1,024$ units and receptive field of size $3 \times 3$ needs $9,216$.

\subsubsection{Pooling Layer}

To prevent overfitting, pooling layers are employed to further distill patterns captured by a CNN.
They progressively reduce the size of the features, leading to smaller number of model parameters and reduced computational time [].

A pooling layer is typically placed after a convolutional one, iterating its activation maps and systematically applying a specific operation over its receptive field.
We differentiate between two primary types of pooling: \emph{max} and \emph{average}.
As their names imply, max pooling selects the maximum value within its receptive field, while average pooling computes the mean of the values.
Pooling layer has its own filters and strides, however, the parameters are chosen more conservatively.
A common choice is $2 \times 2$ filter and stride of $2$, ensuring that the pooled sectors do not overlap.
According to the observations of [https://arxiv.org/pdf/1511.08458], having a filter of size greater than $3 \times 3$ will most likely lead to a loss of the model's performance since such granularity may hide too many of the features detected by a network [].
\todo{citation}

\begin{figure}
    \begin{center}
    \begin{minipage}{0.5\textwidth}
      \includegraphics[width=\textwidth]{img/cnn-pool.png}
    \end{minipage}
    \caption{Example of both max and average pooling layers. Both have filters of size $2x2$ and stride set to $2$, resulting in no overlap during computation. Notice, that since each $2x2$ region is mapped to a single value, the new activation mask has quarter the features of the original.}
    \label{fig:cnn-pooling}
    \end{center}
\end{figure}

\subsubsection{Global Pooling Layer}

Given our network composed solely of convolutional and pooling layers, we are not bound to any specific input size -- thanks to the inherent spatial invariance.
Nonetheless, it is common practice to introduce fully connected layers towards the end of the network, which require a fixed-size number of input features.
To meet this requirement, we use global pooling layers.

Similar to standard pooling, global pooling comes in two forms: max and average.
Global pooling layers operates on each activation map, condensing it into a single value.
This guarantees that if a convolutional layer has $n$ units and the subsequent fully-connected layer receives $n$ input features, placing a global pooling layer in between will provide a compatible transition, standardizing the output size regardless of the original input dimensions.

Throughout this thesis, we will denote the final value of globally pooled activation map $A^k$ as $a^k$.

  \chapter{Prostate Cancer Detection Setting}

According to the Masaryk Memorial Cancer Institute \cite{mmci-prostate-cancer}, prostate cancer is the most common oncological disease among men in the Czech Republic, with approximately $8$ thousand new cases reported each year.
In the global context, experts estimate $1,276,106$ of new cases appeared in $2018$ alone, as reported in \cite{world-prostate-cancer}.
To aid pathologists in tackling the ever-growing number of new cases, the RationAI team trained and tested a CNN on a dataset provided by Masaryk Memorial Cancer Institute.

This chapter provides a brief introduction to prostate cancer and the contemporary approach to detecting malignant tissue, which makes CNN's suitable architecture for assisted diagnosis.
Additionally, we present a dataset and RationAI’s model designed for prostate cancer detection.

\section{Prostate Cancer}

Prostate cancer is a disease that causes rapid growth of cells in the prostate -- a male gland located under the bladder.
This gland produces seminal fluid, which helps transport and nourish sperm.
The type of cancer that attacks the glandular tissue is called adenocarcinoma.

According to \cite{world-prostate-cancer}, doctors commonly diagnose prostate cancer in men after the age of $50$, and the incidence rate keeps increasing with age --- we detect nearly $60$\% presence in men over $65$.
The mortality rate per $100,000$ people varies worldwide, ranging from $3.3$ in East Asia to $10.7$ in Central America. Diet, physical activity, ethnicity, and family history are all likely to influence the development and progression of cancer.
Implications of prostate cancer are significant even if it does not result in death, as it affects one's quality of life due to potential urinary, bowel, and sexual dysfunctions.
In advanced stages, prostate cancer can spread beyond the prostate gland and affect the bladder, rectum, or bones.
Given its high prevalence and potential for aggressiveness, early detection and effective treatment of prostate cancer are vital to improve outcomes and survival rates.

\subsection*{Gleason Patterns and Score}

As Delahunt et al. \cite{gleason-patterns} documented, in 1996, Donald Gleason introduced a unified grading and scoring system to effectively detect and assess the impact of adenocarcinoma spread in the prostate.
Gleason's system gained acceptance in 1974 and is the most prevalent system doctors use today.
It categorizes the growth of cancer cells into distinct patterns based on how much the cancerous tissue differs from healthy prostate gland cells, as shown in \myref{Figure}{fig:gp}.
Several patterns and their descriptions have since been refined, and we now distinguish between 9 patterns.
A full list and description of the patterns can be found in \cite{gleason-patterns}.

To calculate a Gleason score, the histopathologist determines the predominant and second most common Gleason pattern -- the final score is simply a sum of the pattern category numbers.
The International Society of Urological Pathology distinguishes between $5$ grades of prostate cancer.
For more on scoring and grading innuendos, see \cite{gleason-pattern-grading}.
In 2016, the World Health Organization refined the grading into so-called \emph{grade groups}, summarized in \cite{who-grade-groups}.

As this approach to cancer detection is based on visual differences between cancerous and healthy tissue, cancer detection is an ideal setting for convolutional neural networks.

\begin{figure}
    \centering
    \fcolorbox{RoyalBlue}{white}
      {\includegraphics[width=0.97\textwidth]{img/gp-classification.png}}
    \caption{Example of Gleason Patterns ranked from $1$ to $5$. Pattern $1$ represents a healthy stroma --- ``cells and tissues that support and give structure to organs, glands, or other tissues in the body'', per definition from \cite{nci-stroma}. Pattern $5$ represents tissue with the highest risk of cancer. Sourced from \cite{gleason-pattern-description}.}
    \label{fig:gp}
\end{figure}

\section{Dataset}\label{sec:dataset}

The Masaryk Memorial Cancer Institute provided the RationAI group with a pseudonymized hematoxylin/eosin-stained WSIs dataset.
Each WSI contains $3$ to $5$ biopsies and is stored as an uncompressed PNG of size $105,185 \times 221,772$ pixels.
The dataset is stratified into training and test splits, containing $264$ WSIs with cancer, $436$ without, $37$ WSIs with cancer, and $50$ without, respectively \cite{gallo}.
To mark cancerous areas, domain experts manually annotated areas of WSIs containing adenocarcinomas -- annotations are polygons encapsulating cancerous regions.
\myref{Table}{tab:who_grade_distribution} shows the WHO grade group distribution.

\begin{table}
\centering
\ra{1.2}
\begin{tabular}{@{} l r r @{}}\toprule
WHO Grade Distribution & Training split & Test split \\ 
\midrule
Grade 1         & 38 cases            & 5 cases      \\
Grade 2         & 31 cases            & 1 case       \\
Grade 3         & 16 cases            & 1 case       \\
Grade 4         & 9 cases             & 1 case       \\
Grade 5         & 10 cases            & 2 case       \\
\bottomrule
\end{tabular}
\caption{WHO grade distributions for the training and test splits of the prostate WSI dataset. Each case represents a distinct patient. The test split consists of $87$ biopsies from 10 patients and is intended to ``represent different Gleason patterns
and types of infiltration'' \cite{gallo}, loosely copying the distribution of the training split. While each case is from a patient with a positive diagnosis, some of the biopsies do not contain cancer, so the positive/negative ratio is $37/50$.}
\label{tab:who_grade_distribution}
\end{table}

WSIs come with a so-called pyramid structure, which consists of multiple layers of the same image at different resolutions, each representing a deeper level of magnification.
As the user zooms in on a particular area of the slide, higher resolution layers of the pyramid are rendered, providing more detail without compromising loading times.
Because processing the entire WSI at once is computationally expensive, we tile the WSIs to reduce memory requirements.
Each WSI is divided into overlapping tiles of size $512 \times 512$ pixels.
An overlap of $256$ pixels is used between two consecutive tiles to counteract the possibility of severing critical patterns.
We cut out tiles at a resolution of $0.344$ \si{\micro\meter} per pixel, which corresponds to a $10$ times magnification and sample level of $1$ for our WSIs.

Since WSIs contain multiple biopsies with gaps in between them, we chisel out only relevant parts.
Any tile with less than $50$\% of its area covered by tissue is discarded from the training and test dataset splits.
Consequently, although all WSIs are the same size, the number of usable tiles per slide is different.

\section{Model}\label{model}

A model trained by the RationAI group to perform cancer detection in tiled WSIs is a modified version of the well-known CNN architecture VGG16, introduced in \cite{vgg16}.
\myref{Figure}{fig:rationai-vgg16} shows the full model decomposition.

Loosely paraphrasing \cite{gallo}, the model aims to solve the following problem: 
\emph{
Classify tiles according to the presence/absence of cancer in their central square area of $256 \times 256 $ pixels.
Presence is determined by whether the central square overlaps with the domain expert's annotation.
}
The model performs remarkably well on the test split, reaching $100$\% slide-level prediction accuracy --- that is, the model correctly determines whether a WSI contains cancer for each test WSI.

However, in a high-stakes environment such as cancer detection, it is not enough to blindly trust the model based on accuracy.
To verify that the model bases its decisions on relevant morphological features, Gallo et at. \cite{gallo} performed an experiment, presenting a subset of the dataset from \myref{Section}{sec:dataset} including regions of model's interest annotated by the technique described in \myref{Section}{occlusion} to a domain expert.
The domain expert concluded that the model focuses on the critical features on which a human pathologist would base their decisions.

\begin{figure}
    \begin{center}
    \begin{minipage}{0.75\textwidth}
      \includegraphics[width=\textwidth]{img/nn-arch.png}
    \end{minipage}
    \caption{Architecture of the prostate cancer binary classifier model used by the RationAI group. The model is based on VGG16, introduced by Simonyan and Zisserman in \cite{vgg16}. They originally trained their VGG16 model to classify $1000$ classes from the ILSVRC dataset \cite{ilsvrc} and used three FC layers before applying softmax to the output of the last one. The RationAIs' model places global max pooling before a single FC layer, reducing each of the $512$ activation maps to a single value. A sigmoid is applied to the output of the FC layer to scale the raw output value in the range $(0, 1)$. We can, therefore, interpret the final output as the probability of pro-cancer markers in the central $256 \times 256$ pixel square of a given tile.}
    \label{fig:rationai-vgg16}
    \end{center}
\end{figure}


  \chapter{Explaining Neural Networks}

This chapter provides an introduction to the growing field of explainable artificial intelligence.
We start with the motivation for understanding the decisions of neural networks and a brief overview of the legal obligations imposed on decision support systems by the European Union.
We continue with a review of several explainability methods suitable for our domain and task.
This will serve as a basis for the next chapter, where we evaluate selected methods against our benchmark.

In contemporary literature, several terms are used to address the incomprehensibility of ML models. Arrieta et al. \cite{arrieta-taxonomy} distinguish the following idioms:

\begin{enumerate}
    \item Understandability: Characteristic of a model to make a human understand its function without any need for explaining its internal structure or the algorithmic means by which the model processes data internally
    \item Comprehensibility: Ability of a learning algorithm to represent its learned knowledge in a human-understandable fashion 
    \item Interpretability: Ability to explain or to provide the meaning in understandable terms to a human.
    \item Explainability: An accurate proxy of the decision maker and comprehensible to humans.
    \item Transparency: A model is considered to be transparent if, by itself, it is understandable.
\end{enumerate}

In cite \cite{arrieta-taxonomy}, they further emphasize the distinction between interpretability and explainability -- interpretability, closely coupled with transparency, are inherent and passive properties of a machine learning model.
Explainability, on the other hand, is an initiated action taken to clarify the internal details of the models.
Both can be seen as means to achieve understandability -- how a human can make sense of decisions made by a model.


\section{Need for understandability}\label{sec:need-for-xai}

State-of-the-art neural network models are often products of billions of parameters \cite{arrieta-taxonomy}.
The term "black-box" models has been coined to highlight their complex internal mechanics.
In a crusade for ever greater performance and accuracy, models are inherently growing in size and depth.
According to \cite{arrieta-taxonomy, xai-survey}, this increasing size and complexity raises concerns in the research community and the general public about whether these networks can be trusted and used responsibly.

\subsection*{Spurious Correlations}

Distrust does not only stem from a lack of insight into the model's internal reasoning.
In some cases, the flawless performance of machine learning models may result from a systematic bias in training and evaluation data.
A common example is an experiment by Ribeiro et al. \cite{xai-husky}, where they deliberately trained a classifier to distinguish between wolves and huskies.
They constructed the dataset so that images of wolves consistently featured snow in the background, whereas images of huskies did not.
This led the classifier to base its decisions on the presence of snow in the image rather than the animal itself. 

\myref{Figure}{fig:horse-tag} presents a more unintentional example.
One can see that a \emph{capable} model can arrive at the desired output despite ignoring features on which humans would base their decision-making process. 

Examples like this show that the understandability of a decision support system is not just an issue for the end user.
It's also beneficial to the model's development cycle.
By gaining insight into the factors that influence the model's decisions, we can responsibly assess whether its reasoning matches our expectations --- and take action if it does not.

\begin{figure}[!h]
    \begin{center}
    \begin{minipage}{1\textwidth}
      \includegraphics[width=\textwidth]{img/horse-tag.png}
    \end{minipage}
    \caption{Experiment conducted by Samek et al. \cite{xai-horse}. Their model was trained on the PASCAL-VOC dataset \cite{pascal-voc}. After the training, they found the presence of so-called \emph{spurious correlation}. Pictures of horses are classified solely on whether the bottom-left corner of an input image contains a source tag. If the source tag is manually added to an otherwise correctly classified image of a car --- the model changes its prediction, and the image is classified as a horse instead. Picture taken from \cite{xai-horse}.}
    \label{fig:horse-tag}
    \end{center}

\end{figure}

\subsection*{Explainability in medical domain}

Such spurious correlations and the black-box tag make clinicians skeptical about integrating DL models into healthcare.
A recent survey by GE HealthCare \cite{ge-healthcare-survey} concludes that of $2000$ clinicians who participated, $58$ percent do not have overall trust in artificial intelligence systems, and $44$ percent of respondents believe that the AI-based systems are biased.
Gaining the trust of both clinicians and patients is crucial to enabling industry-wide utilization of deep learning-based decision support systems.

\subsection*{Legal obligations for AI explainability in the European Union}

In certain contexts, understanding the model is not only a moral obligation.
In 2016, the European Union included the \emph{right to explanation} within the General Data Protection Regulation (GDPR).
Specifically, Articles 13 and 14 of the GDPR give individuals the right to receive ``meaningful information about the logic involved'' when they are a part of an automatic decision-making process.
This implies that we must take adequate measures to be able to explain decisions made by deep learning-based DSSs.
More on what Articles 13 and 14 mean for AI is to be found in a paper by Goodman and Flaxman \cite{xai-gdpr}.

Another piece of legislation from the European Union is anticipated to be implemented in May or June of 2024.
The \emph{AI Act} introduces a series of regulations for machine learning-based AI systems.
The legislation is notably more detailed and exhaustive than Articles 13 and 14 of the GDPR, yielding mixed responses from domain experts.
While the full implications are yet to be seen, it already imposes several restrictions on what must be met before deploying machine learning models.
One section of the legislation specifically targets high-risk AI systems.
It states that ``providers must build for human oversight, incorporating human-machine interface tools to ensure systems can be effectively overseen by natural persons.''.
Several aspects of the models' life-cycle are also revised, ranging from training datasets to technical documentation.
A comprehensive overview of the article is beyond the scope of this thesis and is summarized by Veale and Borgesius in \cite{xai-ai-act}.

\section{Explainable Artificial Intelligence}

The rapid development of deep learning models and scarcity of trust among users, combined with an attempted regulation, necessitate the development of tools and methods that allow us to understand and justify the outputs of otherwise opaque systems.
To shed light on the internal processes of machine learning models, a sub-field of Explainable Artificial Intelligence (XAI) covers a range of techniques to make models more understandable while preserving their performance and effectiveness.
XAI techniques enable humans to build trust and manage machine learning-based decision support systems, a crucial part of integrating systems based on deep learning.
The exact borders of the field are not firmly set, and the notion of what it means to explain a model is an ongoing matter of discussion \cite{xai-survey}.

In this thesis, the term `XAI' will refer to the subfield dedicated to advancing understandability in AI, and `explainable artificial intelligence' will denote the attribute of machine learning models that enables a certain level of understandability.
Arietta et al. \cite{arrieta-taxonomy} describe explainable artificial intelligence as follows: ``Given an audience, an explainable Artificial Intelligence is one that produces details or reasons to make its functioning clear or easy to understand.''.

The field of XAI distinguishes between two means of making artificial intelligence understandable:
\begin{enumerate}
    \item Using transparent model: Models based on linear regression or decision trees are inherently interpretable -- therefore understandable by themselves. We can derive how they came to the given conclusion by looking at their parameters.
    \item Using post-hoc explainability methods: When dealing with neural networks, we gain little insight into how they operate by simply inspecting their weights. Therefore, we must implement auxiliary methods, which simplify and distill the reasoning of networks so that, according to the definition --- we get a clear and easy-to-understand explanation. Post-hoc explainability methods can be further divided into two subgroups -- global methods explaining the model as a whole and local methods, which focus on internal reasoning behind prediction for a particular input sample.
\end{enumerate}

In recent years, a plethora of both global and local post-hoc explainability methods have been introduced to help grasp the decision process of deep neural networks.
According to the taxonomy proposed by Arietta et al. \cite{arrieta-taxonomy}, post-hoc methods can be further divided into two groups. 
Model-specific methods are designed to explain only models with specific features and capabilities -- such as attention-based models or convolutional neural networks.
Conversely, model-agnostic methods are capable of explaining arbitrary machine-learning models.

\section{Making CNN's Understandable}\label{sec:xai-cnn}
\todo{what is attribution}

For convolutional neural networks, the standard is to visually highlight parts of an image that are considered \emph{important} to the model.
The result of such visualization is commonly referred to as a saliency map, class activation map, or attribution map.
There is no consensus on terminology, and we will use the terms saliency map and explanation to describe any heatmap that identifies regions considered important by a post-hoc method.

The following subsections overview several explainability methods that we benchmark in \myref{Chapter}{experiment}.
We deliberately chose methods not covered in \cite{gallo}.
Methods such as LIME \cite{xai-husky}, Deconvolution \cite{deconvolution}, or $\epsilon$-LRP \cite{lrp}  were either computationally too expensive or produced sub-par results, which are not deemed satisfactory \cite{gallo}.
When choosing methods for this thesis, we focused on the observation of Gallo et al. \cite{gallo} --- methods, which tend to attribute continuous regions of tiles scored higher in the previous benchmarks.
Therefore, we focused on methods specifically created for explaining CNNs, leveraging their spatial memory ability, some of which were overviewed in \cite{krajnansky-grad-cam, bajger-grad-cam, hruska-grad-cam}.

\subsection{Occlusion}\label{occlusion}

Occlusion is a model-agnostic method from a family of so-called input perturbation-based methods.
Occlusion computes the saliency map by systematically covering square parts in the input image and observing the change in models' prediction confidence on the perturbed image.
Intuitively, we expect that when we cover (occlude) an important part of the input image, the confidence of our classifier drops.
On the contrary, when we cover a region without relevant features, the output should not differ too much from the output for the non-perturbed image.
The visual example can be seen in \myref{Figure}{fig:occ-saliency}. This approach gives us a rough (depending on the patch size) heatmap of input feature saliency.

An experiment conducted by Gallo et al. in \cite{gallo} shows that this method produces semantically correct saliency maps, recognizing morphological features indicating several Gleason patterns.
Sadly, Occlusion comes with high computational complexity and resource utilization.
For our use case, using an occlusion patch of $55 \times 55$ pixels and stride of $27$ pixels, the method needs 289 forward passes to compute the saliency map for a single tile.
On a machine with $8$ cores, $16$GB of RAM, and $48$GB of GPU memory capacity, occlusion needs approx. $2$ seconds and $40$GB of GPU memory to generate a single saliency map.
Recall \myref{Section}{sec:dataset} and our approach to WSI tiling.
In our test set, slide size varies from 400 to 4000 tiles per slide.
Omitting auxiliary saliency map processing, occlusion takes anywhere from $13$ to $130$ minutes to explain a single WSI.

\begin{figure}[!h]
    \begin{center}
    \begin{minipage}{1\textwidth}
      \includegraphics[width=\textwidth]{img/occlusion.png}
    \end{minipage}
    \caption{Sample of prostate tissue with Occlusion generated saliency. Green areas denote parts important for the model, while red parts signify non-cancerous areas. Since saliency maps are generated on a tile level, they are further composed, and overlapping areas are averaged to produce slide-level saliency maps. We apply the sigmoid function to saliency maps before combining, as is reported in \cite{gallo} to positively affect the smoothing and stability of the explanations. We do not use produced saliency maps as they are since they tend to be very noisy, as depicted in the left picture. Instead, the saliency map is thresholded to include only the most salient features, as depicted on the right, with a threshold of $55$ percent.}
    \label{fig:occ-saliency}
    \end{center}
\end{figure}

\subsection{CAM \& GradCAM}\label{subsec:cam}

Zhout et al. \cite{cam} show that introducing a GAP layer to a convolutional neural network has a welcome side-effect --- aside from regularization during training, it enhances network localization capabilities, despite being trained on image-level labels only --- without specifying where in the input the object of interest resides.
Their proposed method, called Class Activation Mapping (CAM), is used to visually highlight regions of input image used by the CNN to predict certain class $c$.
This model-specific method works only with networks with a GAP (or GMP) layer as the intermediate between the last convolutional layer and the final fully connected layer.
If a network satisfies this property, it is said to have a ``CAM architecture''.
The original paper uses a network with a convolutional layer before global pooling.
Our model features there a pooling layer instead.
This difference does not affect the results overviewed in the following paragraphs, and we will model them using a convolutional layer as per the original paper. 

Intuitively, CAM is a weighted sum of activation maps of the last convolutional layer.
Consider a network with convolutional layer with $K$ activation maps $A^1, A^2, \ldots, A^K$, followed by a GAP layer and a single fully connected layer.
In the fully connected layer, we fix neuron denoting class $c$, for which we want to compute a saliency map.
Recall \myref{Equation}{gap}, and that GAP reduces the value of each activation map $A^k$ to a single value $a^k$ --- the average of the activation map.
The score of the network for class $c$ is then calculated as
\begin{equation}\label{eq:gap-score}
    y^c = \sum_{k=1}^K w_{ck} a^k.
\end{equation}

To construct a CAM saliency map for class $c$, we reverse-engineer the calculation of $y^c$. The saliency map is constructed out of individual mappings, and mapping for spatial element $(S^c_{\text{CAM}})_{ij}$ is calculated as a weighted sum of activation maps $A^k$
\begin{equation}\label{eq:cam}
    (S^c_{\text{CAM}})_{ij} = \sum_{k=1}^K w_{ck}  A^k_{ij}.
\end{equation}
To obtain saliency map $S^c_{\text{CAM}}$, we calculate mappings for all pairs $i, j$.
The saliency map $S^c_{CAM}$ directly indicates the importance of the respective spatial locations.
If pooling layers are present in the network, they lead to the last convolutional layer having activation maps of smaller dimensions than the input grid.
This is addressed by up-sampling the $S^c_{\text{CAM}}$ to the size of the original input image.

The authors point out that this method is particularly useful for networks featuring the GAP layer.
They claim that using GMP instead leads to the method pointing out the single most discriminatory location instead of all of them, as is the case for GAP.
While this has been largely confirmed by their experiment on the ILSVRC dataset \cite{ilsvrc}, we believe that the method is worth considering for our use case.
Models employed in the experiment by Zhou et al. \cite{cam} are trained to distinguish between $1000$ classes, with their last convolutional layer having $1024$ units.
This configuration suggests a roughly one-to-one relationship between the pooled activations and the classes the network aims to identify.
In our setting, the last convolutional layer has $512$ units to identify one class --- we posit that even though we eventually only extract the most discriminative location from each activation map, having $512$ units should still yield robust localization performance.
Our belief is supported by the fact that upon inspection of the weights of the fully connected layer of the model described in \myref{Section}{model}, $267$ weights are positive --- meaning corresponding pooled activations are pro-cancerous.

Unlike CAM, GradCAM \cite{grad-cam} is a model-agnostic method that can explain various CNN architectures without needing a global pooling layer.
The idea is that since the convolutional layer holds spatial information about its activations, we can highlight the important ones using partial derivatives of the score $y^c$.

We will use the same setting as for \myref{Equation}{eq:cam}, with $L$ being a fixed convolutional layer of our network with $K$ units whose activation maps are $A^1, A^2, ..., A^K$.
GradCAM computes importance weight $i_{ck}$ for feature map $A^k$ as the average of partial derivatives of $y^c$ with respect to individual activations of $A^k$ \cite{grad-cam}
\begin{equation}\label{grad-cam-weights}
    i_{ck} = \frac{1}{|A^k|} \sum_{i=0}^H \sum_{j=0}^C \frac{\partial y^c}{\partial A^k_{ij}}.
\end{equation}

We substitute $i_{ck}$ for $w_{ck}$ in \myref{Equation}{eq:cam} and compute a weighted combination of activations and their respective importance to obtain a saliency map.
$\operatorname{ReLU}$ is applied over the weighted activations to get only positive values advocating for important features with respect to class $c$ \cite{grad-cam}
\begin{equation}\label{eq:gradcam-saliency-map}
    (S^c_{\text{GradCAM}})_{ij} = \operatorname{ReLU}\biggl(\sum_{k=1}^K i_{ck} A^k_{ij}\biggr).
\end{equation}
Authors of GradCAM took the inspiration for using $\relu$ from Deconvolution and Guided Backpropagation, both post-hoc methods already covered in \cite{gallo}.
In case the saliency map $S^c_{\text{GradCAM}}$ is smaller than the input grid, we up-sample it as in the case of $S^c_{\text{CAM}}$.

As shown in \cite{bajger-grad-cam}, for our model, GradCAM saliency maps are identical to the ones generated by CAM, up to a constant factor of ${1}/{|A^k|}$.
This consistency arises from our choice of GMP as the global pooling layer, which renders all partial derivatives zero, except for the derivative corresponding to the maximum activation in a given activation map $A^k$, which thus equals~$w_{ck}$.\looseness=-1

GradCAM has been previously covered in \cite{hruska-grad-cam, krajnansky-grad-cam, bajger-grad-cam}, but as far as we are concerned, this method was not evaluated using a quantitative benchmark similar to one from \cite{gallo}.
Since CAM produces saliency maps identical to GradCAM, we decided to include CAM in our benchmarks.

\subsection{GradCAM++}\label{sub:gradcampp}

GradCAM++ \cite{grad-cam-pp} is another gradient-based model-specific method to spatially attribute convolutional layer features for a given class.
Like GradCAM, this method leverages gradients flowing through the network to assess individual activations' importance while adressing several of GradCAMs' imperfections.

Chattopadhyay et al. \cite{grad-cam-pp} observed that if multiple occurrences of a class of interest $c$ are present in the input image, the localization capabilities of GradCAM worsen.
According to the authors, this stems from using unweighted partial derivatives in \myref{Equation}{grad-cam-weights}.
If there are multiple occurrences of class $c$, different feature maps may get activated, resulting in a diluted and faded saliency map.
They proposed a generalized solution that is equivalent in terms of computational performance.
First, they encode the importance weight $i_{ck}$ from \myref{Equation}{grad-cam-weights} as
\begin{equation}\label{eq:grad-cam-pp-weights}
    i_{ck} = \sum_i^H \sum_j^W \alpha^{ck}_{ij} \relu\biggl(\frac{\partial Y^c}{\partial A^k_{ij}}\biggr)
\end{equation}
adding a weight to each of the positive partial derivatives.
Here, $Y^c$ denotes the score $y^c$ put through the exponential function.

Then, they derive a method for expressing the partial derivative weights $a^{ck}_{ij}$ for all spatial locations in all activation maps.
They substitute weights from fully connected layer in \myref{Equation}{eq:gap-score} for importance weights $i_{ck}$ from \myref{Equation}{eq:grad-cam-pp-weights}, yielding
\begin{equation}\label{eq:grad-cam-pp-score}
    Y^c = \sum_{k=1}^K \biggl( \sum_{i=0}^H \sum_{j=0}^W \alpha^{ck}_{ij} \relu \bigl(\frac{\partial Y^c}{\partial A^k_{ij}}\bigr) \biggr) \biggl( \sum_{a=0}^H \sum_{b=0}^W A^{k}_{ab} \biggr)
\end{equation}
To express the weight $\alpha^{ck}_{ij}$, we need to take a partial derivative w.r.t. $A^k_{ij}$ twice. After that, we can rearrange the terms to obtain
\begin{equation}
    \alpha^{ck}_{ij} = \frac{\frac{\partial^2 Y^c}{(\partial A^k_{ij})^2}}{2\frac{\partial^2 Y^c}{(\partial A^k_{ij})^2} + \sum_{a=0}^H \sum_{b=0}^W A^{k}_{ab} \bigl( \frac{\partial^3 Y^c}{(\partial A^k_{ij})^3} \bigr)}.
\end{equation}

All that is left is to plug this expression back into \myref{Equation}{eq:grad-cam-pp-weights}.
This gives us the importance weights for activations, and the saliency map $S^c_{\text{GradCAM++}}$ is computed the same way as $S^c_{\text{GradCAM}}$.
Thanks to the additional coefficient $\alpha^{ck}_{ij}$ for respective partial derivative, authors claim that all spatial coordinates corresponding to instances of class $c$ in the input data will be highlighted with equal importance in the resulting saliency map.

As shown in \myref{Subsection}{subsec:cam}, two different methods can produce the same saliency map.
We derive the particular values for GradCAM++ importance weights $i_{ck}$ to avoid adding a third one.
The full derivative is presented in \myref{Section}{sec:grad-cam-plus-plus-weight-derivation} and yields
\begin{equation}
    i_{ck} = \frac{1}{2 + A^k_{ij}w_{ck}} \relu\bigl(\exp(y^c)w_{ck}\bigr)
\end{equation}
for $A^k_{ij}$ being the highest activation in activation map of unit $k$.

As a result, only activation maps corresponding to positive weights --- therefore, patterns detected as pro-cancerous --- are present in the final saliency map.
We lack rationale about the term $A^k_{ij} w_{ck}$ in the denominator, as more profound activations will receive a smaller importance weight.
Nevertheless, we have shown that for our architecture, GradCAM++ produces saliency maps that differ from CAM and GradCAM.
The extent to which the activation maps will differ hard to estimate, and we will include GradCAM++ in evaluation in \myref{Chapter}{experiment}.

\subsection{HiResCAM}\label{sub:hirescam}

HiResCAM, introduced in \cite{hires-cam}, is another gradient-based model-agnostic method that builds upon GradCAM's success.
The first step is the same as for GradCAM++ --- we need to compute partial derivatives of $y^c$ with respect to activation map $A^k$.
Instead of condensing the importance of the activation map to a single value, we arrange the partial derivatives into a grid and perform a pairwise multiplication of this grid with the respective activation map.
We compute the grid of partial derivatives for all activation maps $A^k$, and
\begin{equation}
    (S^c_{\text{HiResCAM}})_{ij}
        = \sum_{k=1}^K \frac{\partial y^c}{\partial A^k_{ij}} A^k_{ij}.
\end{equation}
The definition of the saliency map in \cite{hires-cam} omits the $\relu$, unlike GradCAM and GradCAM++.
However, in the implementation\footnote{\url{https://github.com/rachellea/hirescam}} corresponding to \cite{hires-cam}, the authors include the clipping and scaling --- therefore, HiResCAM saliency maps only include positive saliency.

Draelos and Carin \cite{hires-cam} arrived at a similar observation as in the case of GradCAM++ that GAP-ing the partial derivatives in \myref{Equation}{grad-cam-weights} leads to blurred feature maps. 
They believe that taking element-wise multiplication better reflects how models ``see" the input image.
This way, individual activations are scaled according to their partial derivative before being channel-wise summed up to form the final saliency map.

In addition to the method, the authors present proof of HiResCAM's capabilities.
The proof demonstrates that for convolutional networks ending in one fully connected layer $L$, HiResCAM guarantees to visually highlight all parts of the input image that increase class score $y^c$, given we compute the saliency map for the convolutional layer preceding $L$ \cite{hires-cam}.
Unfortunately, the proof does not cover our model introduced in \myref{Section}{model}.
While the model ends in one fully connected layer, it is separated from the last convolutional layer by an intermediate global max pooling operation, which reduces each feature map to a single value.
This breaks down the proof's assumption that we can extract spatial information from the feature maps in the FC layer, since such information will be destroyed in the GMP operation. 

The authors also show that given a model with CAM architecture employing GAP, saliency maps generated by HiResCAM collapse maps of the CAM method.
This does not hold for models using the GMP layer.
For our model, because of GMP, the partial derivative of $y^c$ with respect to $A^k_{ij}$ will be $w_{ck}$ iff $A^k_{ij}$ is the largest activation in $A^k$, and zero otherwise.
Therefore, we compose the resulting saliency map only from the most significant activations in each unit.
This differs from saliency maps produced by CAM and GradCAM++, which do not discriminate against individual activations that do not contribute to the model's output but are evidence for a detected pattern.
We believe that HiResCAM could better resemble how the Occlusion method produces the saliency maps since occluding features, albeit detected but not contributing to the output score because of GMP, should not produce a high saliency for the given region.

\subsection{ScoreCAM}

% \todo{add what is the baseline}
To bridge the gap between perturbation-based and gradient-based methods, Wang and Wang \cite{score-cam} introduced ScoreCAM to tackle GradCAM's flaws from different perspectives.
Unlike it is the case for previous CAM-based methods, ScoreCAM does not rely on partial derivatives to compute the saliency.
Instead, authors rely on a so-called increase in confidence, which is, given input $x$, baseline $x_b$, and activation $A^k$, defined as:
\begin{equation}
    c_k = f\bigl(x \odot \operatorname{upscale}(A^k)\bigr) - f(x_b)
\end{equation}
where $\operatorname{upscale}$ is a function that resizes activation map $A^k$ to the input grid size and scales it to range $[0, 1]$.
We get a score indicating how much network confidence changes if we only supply areas of the image where filter $F^k$ found a pattern it is trained to detect.
We calculate the final saliency map $S^c_{\text{ScoreCAM}}$ using the respective increases in confidence $c_1, \ldots, c_K$ as importance weights for the $K$ activation maps of layer of interest, replacing corresponding $i_{c1}, \ldots, i_{cK}$ in \myref{Equation}{eq:gradcam-saliency-map}.

\subsection{AblationCAM}

Another gradient-free method, AblationCAM, introduced in \cite{ablation-cam}, utilizes ablation analysis to compute the importance of activation maps.
AblationCAM computes how an individual activation map $A^k$ contributes to the final score $y^c$ by "removing" it from the score computation and observing a drop in the model's confidence.
The influence of the removal of activation map $A^k$ to the final score is defined as
\begin{equation}\label{eq:ablation-cam-importance-weight}
    d_{ck} = \frac{y^c - y^c_k}{y^c} = \frac{w_{ck}a^k}{y^c}
\end{equation}
where $y^c_k$ represents an output of the model, for which the activation $A^k$ is zeroed out.

Similar to Occlusion and input features, if we remove activations important to the network when assessing the presence of class $c$, the score $y^c$ should drop. To obtain the final saliency map, we compute the ablation weights for the layer of interest and substitute them instead of importance weights into \myref{Equation}{eq:gradcam-saliency-map}.

The globally pooled activations in the ablation weights could better distribute the saliency across the WSI once the tile-level explanations are combined, as not only the weight but also the strength of activations is taken into account.
However, we believe this effect is counteracted by the model's confidence $y^c$ in the denominator, which is inversely proportional to individual ablation weights.

\subsection{Layer-Wise Relevance Propagation}

While Occlusion estimated feature importance by observing changes in output and CAM-based methods utilized spatial information in the convolutional layers, Layer-Wise Relevance Propagation (LRP) \cite{lrp} takes a different approach.
Given an output score of the network, LRP utilizes several types of propagation rules to redistribute the score from the output layer down to the input pixels.
The idea is that we can look at how individual features contribute to the output of the network --- each feature value is weighted and sent to the upper layer across the whole network, ultimately contributing to the final class score.
In a reverse fashion, we can compute the importance of the input pixels \cite{lrp}.

To describe this procedure, assume that $i$ and $j$ are two neurons in consecutive layers, so there is a connection from $i$ to $j$. 
The process of propagating relevance $R_j$ from $j$ to $i$ is captured by the following generic propagation rule \cite{lrp}
\begin{equation}
    R_i = \sum_j \frac{z_{ji}}{\sum_k z_{jk}} R_j
\end{equation}
where $z_{ij}$ models to which extent neuron $i$ contributed to neuron $j$'s activation.
The extent $z_{ij}$ is derived by various rules, further described in \cite{lrp}.
The choice of rules is largely left to experimentation.
However, Montavon et al. \cite{lrp} provide a blueprint for choosing layers when working with a VGG-16-based network.

The main building block is the LRP-$0$ rule, where the extent
\begin{equation}
    z_{ji} = y_i w_{ji}.
\end{equation}
Parameters $y_j$ and $w_{ji}$ stand for neurons output and weight, respectively, as defined in \myref{Section}{section:dl}.
% \todo{are those really parameters?}

According to the observation of Montavon et al. \cite{lrp}, using solely LRP-$0$ leads to very noisy explanations.
To tackle the noise, we add a small positive term $\epsilon$ to the denominator --- $\epsilon$ will absorb weak and contradictory contributions, preserving only the salient scores. 
As a result, we typically get less noisy explanations.
Gallo et al. used the LRP-$\epsilon$ rule exclusively in \cite{gallo}.
However, such explanations were still deemed scattered and noisy.
To further enhance produced explanations, we can use the LRP-$\gamma$ rule to favor positive contributions by using a coefficient applied only to the positive weights:
\begin{equation}
    z_{ji} = {a_j \cdot (w_{ji} \cdot \gamma w_{ji}^+)}.
\end{equation}
The $\gamma$ rule draws from the LRP-$\epsilon$ rule and keeps the $\epsilon$ term in the denominator.
Figure \ref{fig:lrp-montavon} shows how multiple rules can be composed to visually enhance explanations.

\begin{figure}[!h]
    \begin{center}
    \begin{minipage}{1\textwidth}
      \includegraphics[width=\textwidth]{img/lrp-montavon.png}
    \end{minipage}
    \caption{Image show explanation results after different combinations of rules are applied. Notice that combining multiple rules yields a visually more coherent saliency map. According to Montavon et al. \cite{lrp}, using LRP-$0$ in the upper layers is beneficial to combat the entanglement of different concepts represented by the network. LRP-$\epsilon$ in the middle layer helps to propagate only the most salient activations, while LRP-$\gamma$ in the lower layer ensures uniform relevance spread. Image taken from \cite{lrp}.}
    \label{fig:lrp-montavon}
    \end{center}
\end{figure}
  \chapter{Suitability Benchmark}\label{experiment}

This chapter consists of four sections. First, we try to outline what a good explanation should fulfill. Then, we tackle the current problem of high computational resource utilization of the current solution. In the third section, we construct a quantitative benchmark, choosing suitable metrics alongside brief reasoning and evidence behind our choice. In the last section, we conduct a subjective evaluation by a domain expert using a previously established approach.

We will use implementation from popular XAI-oriented machine learning libraries, \texttt{captum} (Occlusion, Composite-LRP), \texttt{torch-cam} (CAM) and \texttt{pytorch-grad-cam} (HiResCAM, GradCAM++, ScoreCAM, AblationCAM). All three libraries are publicly available on \texttt{github} and provide convenient interfaces, which we can easily unify in our pipeline.

CAM-based methods can compute saliency maps for arbitrary convolutional pooling layers in our network.
Given the notion of hierarchical learning of representation across the network, attributing the lower layers does not make sense.
While using lower layers 

\section{What makes a good explanation?}

\emph{``Unfortunately, `explainability' is a nebulous and elusive concept that is hard to target.''} \cite{explainability-hard}
\newline

Contemporary research does not provide a unified approach to define the "goodness" of an explanation.
There are attempts to propose a set of properties a good explainability method should fulfill, but they are not aligned \cite{xai-functionality-grounded, explainability-hard, xai-meta-survey, xai-zhou-survey}.
Moreover, some studies present contradictory results, rendering objective conclusions even more challenging \cite{xai-zhou-survey}.
According to Doshi-Velez in \cite{xai-doshi}, "the field is crowded with evaluation methods, and there is no consensus on which are the “right” ones. 
Much less, there is not even agreement on which criteria should be evaluated.".
The challenge lies in the absence of ground truth for produced explanations, and therefore, we have no objective measure of grading the explanations \cite{xai-zhou-survey}.
However, by carefully choosing metrics, we can avoid so-called "anecdotal evidence" and argue that certain explainability methods are sufficient given the audience and use case \cite{xai-anecdotal-evidence}.
We believe that for an explanation to be applicable to our use-case of generating slide-level explanations, it needs to be:
\begin{enumerate}
    \item Performant: This is the main bottleneck of the current solution. We do not care how good a method is without reasonable performance since it cannot be utilized for practical purposes.
    \item Faithful: Ideally, we want to ensure that the explainability method only highlights the relevant parts of the input tile.
    \item Useful: We need the explanations to be presented to domain experts so that it is understandable for them. If they do not understand the explanation, it is of little value. 
\end{enumerate}

\section{Computational performance}

For the method to actively assist pathologists, the explanations must be computed fast enough not to disrupt his workflow.
The main problem of the existing approach using Occlusion-based saliency is to generate explanations for an entire slide in a reasonable time.
The current solution employs \texttt{python} implementation from the \texttt{captum} package.
The problem with Occlusion is two-fold; either we perform one forward pass with the occluded tile at a time, needing XXX synchronous forward passes, or we batch several occluded tiles together, drastically raising GPU memory costs.
To tackle this problem from both perspectives --- we first measure the time required to produce a single explanation and then observe how much GPU memory it needs.

\subsection*{Time efficiency}

 We let each method fully explain the averagely-sized WSI of 1499 tiles from the test set.
 Although we aim to achieve good slide-level performance, we measure the time required for individual tile-level explanations instead.
 During the generation of slide-level explanations, there are a lot of auxiliary operations to stitch tile-level saliency maps together.
 The method and its implementation are not responsible for these operations.
 We include this overhead if we measure the time required to generate slide-level saliency maps.
 Since modern machine-learning frameworks typically employ parallel processing using multiple CPU cores, this can lead to resource locks, negatively affecting a method's performance if we were looking at the required slide-level time.
 Another reason to measure the required time for tile-level explanations is resource saving.
 We want to take multiple unit measurements and see their mean and variance since momentary factors of the machine environment can influence a single data point. 
 By taking multiple data points, we are hedging ourselves against these factors.
 If we were measuring the slide-level performance, it would be notably more time-and-resource intensive, blocking resources from other students and colleagues from the RationAI group.

 We performed each run on a machine with $8$ cores, $16$GB of RAM, and a single NVIDIA A40 GPU card with $48$GB of memory.
 For Occlusion, we used a batch size of $200$ perturbed tiles, the same settings used in the production deployment.
 Other methods do not offer such an option. We repeatedly fed each method a single tile and measured the time difference between the start and end of the computation.
 After the method call, we inserted a \texttt{torch.cuda.synchronize} call to ensure all computation on GPU is finished before taking the end-of-computation timestamp.

\begin{table}
\centering
\ra{1.2}
\begin{tabular}{@{} l r r r @{}}\toprule
Method & Avg. time per tile (s) & Time for slide (s) & Expected largest slide time (m) \\ 
\midrule
CAM           & $0.04 \pm 0.04$ & $59.45$    & $2.73$ \\
GradCAM++     & $0.08 \pm 0.00$ & $131.88$   & $6.05$ \\
HiResCAM      & $0.08 \pm 0.00$ & $131.95$   & $6.06$ \\
Composite-LRP & $0.11 \pm 0.00$ & $178.829$  & $8.21$ \\
\textbf{Occlusion}     & $1.06 \pm 0.00$ & $1612.70$  & $74.07$ \\
AblationCAM   & $9.72 \pm 0.00$ & $14583.50$ & $669.82$ \\
ScoreCAM      & $9.76 \pm 0.00$ & $14629.98$ & $671.96$ \\
\bottomrule
\end{tabular}
\caption{
Time efficiency of overviewed XAI methods.
Methods are ordered from the fastest to the slowest.
Entry for Occlusion is in bold to visually delimit the methods into faster and slower than the current solution.
Unsurprisingly, methods that require a single pass to compute activation maps' importance were the fastest, while the methods perturbing either input or model internals scored slower since they require multiple forward passes to estimate the importance.
We could solve this, as in the case of Occlusion, by batching multiple perturbed inputs together, but the libraries do not offer such an option.
Even then, we would arrive at a different problem; such batching requires tremendous GPU capacity.
The third column represents the estimated time to explain the largest slide from the test split to give a current worst-case scenario of $4131$ tiles-per-slide outlook.
}
\label{tab:comp-time}
\end{table}
\todo{minutes}

\myref{Table}{tab:comp-time} shows the superiority of CAM methods, which require a single forward pass to compute the saliency maps.
Expectedly, ScoreCAM scored the worst.
This stems from the fact that for each of the $512$ activation maps of the last layer of our model, ScoreCAM runs a separate forward pass with a perturbed input tile.
The problem of multiple forward passes is tamed by batching in the case of Occlusion, where most of the overhead comes from the creation of perturbed tiles.
What surprised us was the time AblationCAM needed to compute a single explanation.
\myref{Equation}{eq:ablation-cam-importance-weight} shows that the importance weights in the case of AblationCAM are computed by perturbing models internals --- systematically zeroing out activations across layer units.
In the case of our model, computing importance weight for unit $k$ means setting the pooled activation $a^k$ to zero --- output $y^c_k$ is simply a linear combination of pooled activations, omitting the $a^k$, which we would expect to be computed reasonably fast.

Since both AblationCAM and ScoreCAM performed notably worse than Occlusion and did not bring us closer to finding a performant XAI method, we decided not to include them in further benchmarks.

\subsection*{GPU utilization}

Modern GPUs offer the possibility of simultaneous execution of multiple processes on the same card instance.
The demand a method places on GPU resources is crucial for production deployment.
High GPU utilization by one method can limit the number of processes that can run in parallel, which, in turn, may increase deployment costs due to the need for additional GPU cards.
Since neural networks are generally expensive in terms of computational power, their usage inevitably leads to increased carbon emissions.
Sustainability, nowadays a crucial concern in technology development and deployment, should be thoughtfully weighed when considering the future use and advancement of deep learning models.
We are not aware of any previous work trying to focus on GPU utilization of explainability methods, but given EU regulations on AI from \myref{Section}{sec:need-for-xai}, we believe that it should be evaluated accordingly.
Not considering those requirements could negatively affect deployed models since being unable to facilitate given restrictions may require them to be brought down.

Like the Time Efficiency metric, we let each method explain a single WSI of $1499$ tiles.
We monitor the GPU usage by running the \texttt{nvidia-smi} executable in a separate process, forwarding its output every $500$ milliseconds to a CSV file.
\myref{Table}{tab:gpu-util} presents the utilization of a single A40 GPU card, while computing the explanations.
Note that we cannot exactly distinguish which portion of the GPU memory the XAI method uses since we also store the model on GPU.
For this reason, we conducted one run without computing any explanations to serve as a baseline.
We did all of the runs using a batch size of one in an attempt not to discriminate any method.
We expect all methods to score better than Occlusion since none relies on internal batching.

The results in \myref{Table}{tab:gpu-util} confirm our expectations that none of the methods except Occlusion noticeably exhaust the GPU.
Therefore, we can run multiple processes simultaneously on a single card instance.
Moreover, all methods but Occlusion can be sufficiently run on the smallest GPU the RationAI group has at hand, having a capacity of $10$GB.
\begin{table}
\centering
\ra{1.2}
\begin{tabular}{@{} l r r @{}}\toprule
Method & Absolute GPU utilization (MB) & Relative Overhead (MB) \\ 
\midrule
Vanilla run             & $910$      & -       \\
CAM                     & $1084$     & $174$   \\
GradCAM++               & $1578$     & $668$   \\
HiResCAM                & $1578$     & $668$   \\
Composite-LRP           & $2476$     & $1566$  \\
\textbf{Occlusion}      & $36308$    & $35398$ \\
\bottomrule
\end{tabular}
\caption{
GPU utilization of overviewed XAI methods. We use mode instead of mean, as we observed that after two GPU memory utilization snapshots, the number stays the same up until the end of the benchmark. Column Absolute GPU utilization captures the raw output of \texttt{nvidia-smi} command, while column Relative Overhead corresponds to the additional consumed memory on top of the normal model's requirements. All values are reported in megabytes. All methods scored notably better than Occlusion, which we attribute mainly to not running batched forward passes. CAM achieved the lowest utilization since it only multiplies activation maps with FC layer weights. GradCAM++ and HiResCAM need to compute and retain certain gradients, which require more memory than vanilla pass. Composite-LRP runs one full backward pass equivalent and needs to retain forward pass activations to compute neuron relevance.
}
\label{tab:gpu-util}
\end{table}


\section{Quantitative evaluation}

This section covers several entrenched metrics that capture our desired traits of explainability methods.
We build on the work of \cite{gallo} by incorporating state-of-the-art metrics that address the limitations of previous approaches.
Refining previously used metrics aims to provide a more robust and reliable benchmark for evaluating our methods, aligning the quantitative assessment more closely with domain-specific boundaries and expectations.

\subsection*{Faithfulness}

An established approach to assess whether the explainability method points to the relevant part of the input is to perturb the image, such that we remove features perceived as important by the model and back-fill removed areas with certain fixed value --- similar to how occlusion estimates feature importance --- and observe how model's confidence changes.
However, the literature suggests that this perturbation approach, which involves filling the removed areas with zeros or the mean pixel value, could be flawed.
The perturbed input might include artifacts that induce a shift in the data distribution, compromising the reliability of such metrics since we cannot tell to what extent the change in the model's confidence stems from introduced artifacts compared to the initial relevance of removed features \cite{roar}.

Experiments in \cite{gallo} utilize methods prone to introducing such unintentional artifacts.
Notably, Occlusion and Deep-Taylor Decomposition \cite{xai-dtd} received the highest scores.
This is aligned with the observation by Hsieh et al. in \cite{xai-hsieh-occ-dtd} --- that such metrics may favor methods that rely on the same mechanism when estimating the importance of input features.
Since Occlusion estimates feature importance by a drop in the model's confidence by perturbing the input image with a patch --- when we later remove the region corresponding to a high attribution, the model's score will inevitably drop.
Employed metrics (Causal Deletion \cite{xai-causal-deletion} and Area Over Perturbation Curve \cite{xai-aopc}) also did not produce aligned results, confirming findings in \cite{roar}.

Hooker et al. \cite{roar} introduced an alternative metric known as Remove and Retrain (ROAR), which involves removing (zeroing out) the identified important features and then retraining the model with the modified dataset --- removing the distribution shift.
Although ROAR has gained in popularity, it has the significant downside of being computationally intensive, rendering it unacceptable for our use case.
\todo{maybe say it is shit cause we do not know to what extent the performance deterioration is thanks to bad training and removed data}
As further shown by Rong et al. in \cite{road}, ROAR does not solve the problem of the so-called \emph{class information leakage} --- phenomenon when the uniformly-valued removed region reveals relevant class info through its shape.
Rong et al. propose a method built on the foundations of ROAR, called Remove and Debias (ROAD).
Instead of retraining the model, areas after removed features are specifically imputed to reduce the risk of class information leakage \cite{road}. Refer to the original paper for mathematical intrications of information theory behind the imputation.

To evaluate the performance of our methods, ROAD iteratively removes features from the most relevant to the least (MoRF order).
After each removal, missing features are imputed using a noisy-linear imputer to reduce class information leakage \cite{road}.
For a single tile, $x$, saliency map $S$ and percentage $p$, the result of this method is a number $d$ corresponding to the drop in confidence of a model computing function $f$ when fed the imputed tile
\begin{equation}
    d = f(x) - f\bigl(\operatorname{perturb}_p(x, S)\bigr).
\end{equation}
The function $\operatorname{perturb}_p$ takes the input image $x$, saliency map $S$, and linearly imputes the areas corresponding to the top $p$ percent of most salient pixels of explanation $S$ \cite{road}.

As in one of the experiments in the original paper, we evaluate the methods at $10, 20, 30, 40$, and $50$ percent of the most salient pixels perturbed.
While other experiments in \cite{road} feature imputation up to $85$ percent, it is computationally not feasible in our setting.
Since our methods tend to cover larger areas of input tiles, imputing $85$ of the most salient pixels takes around \todo{add the final time}.
Moreover, in the production setting, the Occlusion saliency maps are thresholded to display anywhere from $55$ to $75$ of the most salient pixels.
As this thesis aims to compare the given methods against the baseline generated by Occlusion, with all these factors in mind, we decided not to extend the perturbation beyond the $50$ percent of the most salient pixels.

It is desired that the model's score should drop significantly as we start removing the features, and the rate slows down as we approach the higher percentages, signaling that the important parts were indeed removed in the beginning.
We evaluated our model using the \texttt{pytorch-grad-cam} package that provides the \texttt{NoisyLinearImputer} class.
This class is designed to handle the imputation of missing values in the same manner as the original implementation from the ROAD paper.
We did not use the original implementation from \cite{road}, as the interface did not fit into our data-processing pipeline.
\myref{Figure}{fig:road-impute} shows how the \texttt{NoisyLinearImputer} imputes removed areas.
Since this method relies on sensitivity, we will only use positive tiles from the test split of the dataset from \myref{Section}{sec:dataset}.
To see the curve of the drop in the model's confidence, refer to \myref{Figure}{fig:road-curve}.
A boxplot portraying the distribution of our results is presented in \myref{Figure}{fig:road-boxplot}.

Our findings are consistent with the study conducted by Gallo et al. \cite{gallo} --- the faithfulness metric we utilized tends to prefer techniques that identify broader, continuous regions within the input tile.
As shown in \myref{Figure}{fig:road-impute}, the tile imputed based on the explanation generated by Occlusion looks visually more distressed than the tile imputed according to the more scattered Composite-LRP method.
This  HiResCAM, which scored worst out of all CAM-based methods since it is the most conservative regarding how much of a tile-produced saliency map covers.
\todo{obrazok alebo to lepsie vysvetlit}
According to \myref{Figure}{fig:road-boxplot}, CAM-based methods produce more consistent results than Occlusion-based saliency.
Similar to $\varepsilon$-LRP in \cite{gallo}, Composite-LRP performs worse than Occlusion, rendering our effort to assign different relevance rules to produce more coherent saliency maps according to \cite{lrp} unsuccessful.
An interesting observation is that upon perturbation based on saliency maps produced by Occlusion and CAM, the models' confidence does not increase, which is not the case for the rest of the methods.
We also posit that despite the authors of GradCAM++ and HiResCAM \cite{grad-cam, hires-cam} advertising their methods as more faithful than CAM/GradCAM, our results suggest otherwise.

\todo{legendu k ciaram}
\begin{figure}
    \begin{center}
    \begin{minipage}{0.7\textwidth}
      \includegraphics[width=\textwidth]{img/road-curve.png}
    \end{minipage}
    \caption{Curve visualizing mean drop in confidence of individual methods per imputed percentages of most salient pixels. Notice that looking only at the mean renders CAM and GradCAM++ as methods achieving similar performance.}
    \label{fig:road-curve}
    \end{center}
\end{figure}

\begin{figure}
    \begin{center}
    \begin{minipage}{1\textwidth}
      \includegraphics[width=\textwidth]{img/road-boxplot.png}
    \end{minipage}
    \caption{Boxplot of results for different percentages of ROAD method. We used the percentages from \cite{road}. ROAD ranks CAM methods over Occlusion and Composite-LRP. We can see that perturbation by CAM-based method explanations also decreases the model's confidence faster than the Occlusion method --- advocating for the CAM explanations to be more faithful. Scores for CAM methods also have less variance when significant parts of salient areas are removed. Notice the minimum values and that only CAM and Occlusion do not increase models' confidence upon evaluation of perturbed tile, while gradient-based CAMs and Composite-LRP do.}
    \label{fig:road-boxplot}
    \end{center}
\end{figure}

\begin{figure}
    \begin{center}
    \begin{minipage}{0.8\textwidth}
      \includegraphics[width=\textwidth]{img/road-impute.png}
    \end{minipage}
    \caption{Upper row depicts attributions for Occlusion and Composite-LRP for the same tile. The bottom row shows a visualization of the tile imputed based on the $20$ percent of the most salient attribution pixels. While \texttt{NoisyLinearImputer} removes continuous regions based on the Occlusion attribution, the scattered Composite-LRP attribution and following imputation lead to a tile visually very similar to the original one. As a result, methods producing less-scattered explanations received higher scores, in agreement with results from \cite{gallo}. Images featuring attributions are presented in black and white to increase the visual contrast between the tile and the explanation. It is important to acknowledge that ROAD favors methods that cover larger areas of the tile --- because, at the same threshold, larger parts of the tile will be imputed. However, the visualization may be misleading, as LRP positively attributed more than $58$ percent of the displayed tile.}
    \label{fig:road-impute}
    \end{center}
\end{figure}

\subsection*{Localization}

The faithfulness metric tells us how well the explainability method attributes important locations.
In addition, we want to ensure that these locations resemble what pathologists would classify as adenocarcinoma.
To measure how well the output of a given method matches the pathologists' annotation, Gallo et al. \cite{gallo} utilized the Effective Heat Ratio (EHR) \cite{ehr} metric.
EHR relies on ground truth bounding boxes, which encapsulate objects of interest.
Because of our specific use case, we argue that EHR is not the most suitable method.
Given a bounding box, by design, EHR favors methods that cover larger annotation areas with strong saliency.
This is not necessarily desired, as the bounding box only marks the rough location of cancerous tissue --- precise pixel-level borders are difficult to follow \cite{gallo, annotation-agreement} and such precise annotation would likely vary from pathologist to pathologist \cite{annotation-agreement}.
Therefore, EHR discriminates metrics that could potentially point to cancerous markers but do not attribute healthy surrounding tissue included in the annotation.

We will use what we consider a simpler technique called Weighting Game from \cite{weighting-game}, instead.
It builds upon the established Pointing Game metric \cite{pointing-game}, which looks at whether the most salient pixel falls into the bounding box.
Unlike the Pointing Game, which gives us just binary information about the accuracy of a given method, the Weighting Game calculates the ratio of the mass of the saliency map $S$ within the bounding box to the total mass of the $S$ \cite{weighting-game}.
Compared to EHR, this does not disqualify methods that highlight smaller parts of the annotated area, and we believe it gives us a fair measure of how the explanation holds compared to the pathologist's annotation.
Given a tile $x$, annotation in the form of binary mask depicting the annotated region $A$ and saliency map $S$, the ratio $r$ is calculated as
\begin{equation}\label{eq:wg}
    r = \frac{\operatorname{\text{mass}}(S \odot A)}{\operatorname{\text{mass}}(S)}.
\end{equation}
In the Weighting Game paper \cite{weighting-game}, authors first blur the mask a little by convolving it with a $3 \times 3$ filter to counter possible imprecision given a pixel-level ground truth bounding boxes.
We decided to omit this explicit blurring step since, given our domain, the annotations are implicitly not ground truth for exact borders of cancerous tissue, and such imprecisions are already counted for given the more rough polygonal shape.

It is important to understand the role of this metric in our benchmark.
Post-hoc XAI methods serve as a proxy between a model and its user.
The model may have its flaws and hidden biases, and therefore, we cannot disregard a method solely on a bad Weighting Game score.
Such a method may still reveal important information and help to facilitate such flaws and biases.
However, our model is believed to have an excellent performance, rigorously verified with a domain expert in \cite{gallo}.
Therefore, we see this metric as a bridge between the quantitative faithfulness metric and qualitative but inherently subjective domain expert assessment.

\myref{Figure}{fig:weighting-game-boxplot} depicts rations distribution per positive tiles in the test part of our dataset.
Since we threshold Occlusion saliency maps when looking at the WSIs in the browser, we evaluate across different percentages of the most salient pixels kept.
Gradient-based CAMs outperform Occlusion across all thresholds, advocating for the positive influence of gradient on localization capabilities.
CAM performs similarly, slowly catching up as we keep only the most salient attributions.
Notice that the score for Composite-LRP improves very slowly.
This likely means that Composite-LRP tends to strongly attribute features that do not fall within the annotated region, and these attributions are preserved even upon keeping only the very most salient regions.


\begin{figure}
    \begin{center}
    \begin{minipage}{1\textwidth}
      \includegraphics[width=\textwidth]{img/weighting-game-boxplot.png}
    \end{minipage}
    \caption{Boxplot of the Weighting Game rations across different percentages of the most salient pixels kept. Notice the very good localization capabilities of gradient-based CAMs from the very beginning. This is not surprising in the case of HiResCAM since it only averages over the strongest activations, rendering the saliency map more conservative regarding how much of the tile they cover. However, we expected CAM to have similar performance to GradCAM++. T}
    \label{fig:weighting-game-boxplot}
    \end{center}
\end{figure}

\subsection*{Usefulness}

The usefulness of an explanation in the context of deep learning models is inherently subjective, as it can vary significantly depending on the individual's perspective and context --- the audience's background, expertise, and purpose of use all play critical roles in determining the perceived utility of an explanation \cite{xai-doshi} --- something hard to capture by a quantitative metric.

Fortunately, we know that a solution based on Occlusion produces semantically correct saliency maps aligned with features recognized by pathologists \cite{gallo}.
Thus, we will employ the Weighting Game metric to assess how well candidate methods' saliency maps resemble the Occlusion ones.
Therefore, to create an Occlusion-based annotation, we convert the Occlusion saliency map to a binary mask, where positive values get the value of $1$.
In practice, during their analysis, the pathologist adjusts the saliency threshold of the displayed explanations to fall between approximately $55$ to $75$.
Since Occlusion-based annotations cannot be considered a ground truth for relevant morphological features, we will give the candidate methods the benefit of the doubt, creating the annotations at the lower bound of the threshold, $55$ percent.
Then, we use this thresholded binary saliency map as annotation $A$ in \myref{Equation}{eq:wg}.

We do not perceive this metric as adding to candidate methods' faithfulness or localization capabilities.
Instead, its purpose is twofold.
First, it guides our selection of which explainability methods to present to a pathologist, allowing us to prioritize explanations that align most closely with the established Occlusion baseline.
Second, it enables us to evaluate how the pathologist subjectively perceives candidate explainability techniques relative to the established understanding and acceptance of Occlusion.

\myref{Figure}{fig:occ-weighting-game-boxplot} depicts the distribution of Weighting Game ratios across different thresholds for candidate saliency maps.
HiResCAM agrees well with Occlusion-based annotation.
This aligns with our a priori expectation from \myref{Subsection}{sub:hirescam} that Occlusion will attribute mostly regions corresponding to the highest activation maps values since only those are taken into account because of GMP.
Despite GradCAM++ and CAM producing visually similar results, GradCAM++ saliency maps tend to hit better areas attributed by Occlusion.
However, CAM starts to catch up towards the higher thresholds for kept pixels.
Composite-LRP consistently performs across all thresholds, supporting the observation that it attributes regions not considered important by other methods.


\begin{figure}
    \begin{center}
    \begin{minipage}{1\textwidth}
      \includegraphics[width=\textwidth]{img/occlusion-weighting-game-boxplot.png}
    \end{minipage}
    \caption{Boxplot of Weighting Game ratios, when we take $55$ percents of the most salient pixels from Occlusion saliency map as the bounding box. Notice that HiResCAM achieves high ratios very consistently. We attribute the high score to the conservativeness of areas covered by its saliency maps and our expectation from \myref{Subsection}{sub:hirescam}. Despite the similar performance of CAM and GradCAM++ in both faithfulness and localization metrics, GradCAM++ tends to better resemble the Occlusion-based annotation, up until taking only the very most salient pixels from respective saliency maps.}
    \label{fig:occ-weighting-game-boxplot}
    \end{center}
\end{figure}

\section{Domain expert assessment}

This section presents a qualitative evaluation of produced explanations by domain experts.
In \cite{gallo}, Gallo et al. first generated saliency maps for all test slides.
Then, they sampled a subset of $461$ regions, where Occlusion suggested an area important for the model when deciding whether to be pro or against the presence of cancer.
Originally, we wanted to present candidate saliency maps for all the $461$ original regions.
However, this would implicitly handicap methods, which focus on different, albeit still relevant, parts of the WSI. 
Another option was to produce saliency maps of candidate methods for the test set and try to sample new areas of interest. However, we could not use the original approach since it also relied on negative attributions, something we do not have for HiResCAM and GradCAM++.
Moreover, such random sampling could also be unfair, as the distribution of different morphological features would likely vary across different samples.
A rigorous and thorough review of detected cancerous markers in hundreds of samples also requires a non-trivial time investment from a domain expert.

With all that in mind, we reviewed the results of our quantitative metrics and decided to reframe the assessment.
We leverage that our domain expert already has prior experience with results based on Occlusion.
Therefore, we generate saliency maps for candidate methods for all test WSIs.
Since the pathologist's time is not to be wasted, we narrowed down the presented methods based on their quantitative results.
We decided to include explanations of two methods, CAM and HiResCAM.
CAM has comparable quantitative results to GradCAM++ while being faster.
Unlike GradCAM++, saliency maps from CAM also contain negative regions, pointing to locations the model perceives as against cancer --- same as Occlusion.
HiResCAM scored notably well when quantitatively comparing marked salient regions to the Occlusion ones.
In this setting, the target audience is an experienced pathologist, MUDr. Rudolf Nenutil, CSc.
He has a long track record of working with the national group, and his expertise was used for evaluation in \cite{gallo}.
To subjectively assess the usefulness of the saliency maps, we give him the following assignment:

\emph{
In the attached URI, you will find a report with all test slides and saliency maps generated by our candidate methods.
Open them in the WSI browser and subjectively assess whether they could be used instead of the Occlusion saliency maps.
As for Occlusion, try to find a suitable threshold such that the saliency points to the relevant morphological features.
Assess whether those methods point to features and patterns you understand.
Saliency maps of CAM contain green and red regions, with the same meaning as in the case of Occlusion --- green areas denote pro-cancerous features, while red areas denote non-cancerous tissue.
Saliency maps for HiresCAM are in yellow, depicting only the pro-cancerous areas.
Please document your findings, noting where the saliency maps align well or poorly with your understanding of the tissue morphology.
}



  \bookmarksetup{startatroot}

  \chapter*{Conclusion}
\addcontentsline{toc}{chapter}{Conclusion}

The main goal of this thesis was to find a suitably fast method that could replace Occlusion.
This body of work builds upon previous research done by the RationAI group in \cite{gallo, bajger-grad-cam, krajnansky-grad-cam, hruska-grad-cam}.
We reviewed the achieved results and focused on previously outlined perspective directions.
To measure the suitability of candidate methods, we designed a comprehensive quantitative benchmark with specific domain requirements in mind.
As a result, we presented two methods that outperformed Occlusion to a domain expert.
He concluded that both methods focus on all relevant morphological features detected by Occlusion.
Thus, we consider CAM in combination with HiResCAM viable alternatives, and to maximize efficiency, we incorporated the custom implementation of both methods to the \texttt{HistoPipe} repository, reducing the time required to generate slide-level saliency maps for all $87$ test WSI's from $3.3$ days to $2$ hours.

\subsection*{Future Work}

Bajger has shown in \cite{bajger-grad-cam} that using clustering methods in combination with CAM/GradCAM gives promising results.
We want to extend his results and try to segment different xPatterns using such clustered activation maps.
Given a dataset compiled by domain experts to resemble different Gleason patterns, we can look at which activation maps have prominent activations when looking at these patterns.
Suppose there is a degree of separation of activations between groups of convolutional units upon seeing different patterns. In this case, we can predict not only the presence of cancerous tissue but also the severity of the patterns detected --- without any additional training or modification to the model architecture. \cite{tmp}
  % \appendix
\renewcommand{\thechapter}{A}
\chapter{Usage of FIPS 140-2 functionality of sec-certs}

\definecolor{codegreen}{rgb}{0,0.6,0}
\definecolor{codegray}{rgb}{0.5,0.5,0.5}
\definecolor{codepurple}{rgb}{0.58,0,0.82}
\definecolor{backcolour}{rgb}{0.95,0.95,0.92}

\lstdefinestyle{mystyle}{
    backgroundcolor=\color{white},   
    commentstyle=\color{codegreen},
    keywordstyle=\color{magenta},
    numberstyle=\tiny\color{codegray},
    stringstyle=\color{codepurple},
    basicstyle=\ttfamily\footnotesize,
    breakatwhitespace=false,         
    breaklines=true,                 
    captionpos=b,                    
    keepspaces=true,                          
    showspaces=false,                
    showstringspaces=false,
    showtabs=false,                  
    tabsize=2
}

\lstset{style=mystyle}

% \section*{Usage of FIPS 140-2 functionality of sec-certs}

The Python module for analysis of FIPS 140-2 certificates, described in the thesis, is integrated in the \texttt{sec-certs} tool~\cite{sec-certs}. Current version of \texttt{sec-certs} acts like a Python library to be used in users' own scripts. 

\section{Installation}

The tool requires \texttt{Python>=3.8} and accessible \texttt{pdftotext} binary. The tool can be set up in a Python virtual environment and \texttt{setup.py}:

\begin{lstlisting}[language=bash]
    $ python3 -m venv venv
    $ source venv/bin/activate
    $ pip install -e .
\end{lstlisting}
This installs all neccessary python libraries needed for \texttt{sec-certs}.

\section{Folder structure}

\texttt{sec-certs} files related to this thesis are the following:
\begin{itemize}
    \item \texttt{sec\_certs} -- folder containing the Python module files.
    \begin{itemize}
        \item \texttt{dataset} -- folder containing the classes used for dataset manipulation
            \begin{itemize}
                \item \texttt{fips.py} -- file containing the \texttt{FIPSDataset} class, used to handle the main dataset.
                \item \texttt{fips\_algorithm.py} -- file containing the \texttt{FIPSAlgorithmDataset} class, used to handle the algorithm implementations dataset.
            \end{itemize}
            Both these classes inherit from the \texttt{Dataset} class in \texttt{dataset.py}.
        \item \texttt{certificate} -- folder containing the classes used for manipulation with certificate objects used in datasets
            \begin{itemize}
                \item \texttt{fips.py} -- file containing the \texttt{FIPSCertificate} class and \texttt{FIPSAlgorithm} as a subclass.
            \end{itemize}
            The \texttt{FIPSCertificate} class inherits from the \texttt{Certificate} class in \texttt{certificate.py}.
        \item \texttt{configuration.py} -- class containing the global config loaded from the \texttt{YAML} file.
        \item \texttt{cert\_rules.py} -- file containing the regular expressions used during the keyword extraction. The variables including \texttt{fips} in their name are used in the functionality described in this thesis.
        \item \texttt{helpers.py} -- some of the helper functions used during the data extraction.
    \end{itemize}
    \item \texttt{examples} -- folder containing the provided entrypoints for \texttt{sec-certs}. Explained in \myref{Section}{entrypoint}.
    \item \texttt{tests} -- folder containing tests to \texttt{sec-certs}. The files related to the FIPS 140-2 functionality contain \texttt{fips} in their name.
\end{itemize}

\section*{Basic usage}

\texttt{sec-certs} is supposed to be used as a python module. We provided examples on the using of the CC and FIPS functionality in the \texttt{examples/} folder. The \texttt{fips\_oop\_demo.py} script provides a CLI and a main entrypoint for the FIPS 140-2 functionality of \texttt{sec-certs}. Further details of FIPS functionality are explained in \myref{Listing}{code}.

\begin{lstlisting}[language=python,caption=Example of using FIPS 140-2 functionality of \texttt{sec-certs}, label={code}]
    """
    Load config YAML file and make it globally available in the 'config' variable. 
    The example config file is available in 'sec-certs/settings.yaml' file. 
    All presented values are required.
    :param filepath     path to config YAML file
    """
    config.load(filepath="path/to/config/file")
    """
    Create empty dataset object. 
    The 'FIPSDataset' class is implemented in 'sec-certs/dataset/fips.py'.
    :param certs        Dictionary containing <certificate_id, 'FIPSCertificate'> pairs.
    :param root_dir     Path to the directory where all the metadata will be saved. 
    :param name         Name of the dataset.
    :param description  Description of the dataset.
    """
    dataset = FIPSDataset(certs={}, root_dir="path/to/root/dataset/directory", 
                        name='dataset_name', description='dataset description')
    """
    Extracts the information from CMVP web pages and populates the dataset.
    This method provides the 'web_scan' functionality:
        -   downloads the list of CMVP validated modules from the search pages
        -   creates dataset entries
        -   downloads HTML page and security policy document for each available module
        -   extracts information from HTML pages
    :param redo         Extract the information from webpages from scratch.
    :param json_file    Optional - path to saved json file created during previous calculations. 
                        If provided, the dataset is loaded from this file and is only updated.
                        If not provided, the dataset is created all over again.
                        Default value is "dataset.root_dir / 'fips_full_dataset.json'"
    :param test         Optional - provides path to modified CMVP search page, used only in tests to check whether the functionality is correct for a subset of all certificates.
    :param update_json  Optional - if true, the dataset in "dataset.root_dir / 'fips_full_dataset.json'" is updated at the end of this method.
                        If False or not provided, the saved dataset is not updated.                    
    """
    dataset.get_certs_from_web(redo, json_file, test, update_json)
    """
    This method uses pdftotext to convert the security policy PDFs to text files.
    :param update_json  Same functionality as in 'get_certs_from_web()' method.
    """
    dataset.convert_all_pdfs(update_json)
    """
    Extracts the information from converted security policy documents using regular 
    expressions.
    This method provides the 'pdf_scan' functionality:
        -   extracts non-specific keywords used in further analyses
        -   extracts FIPS 140-2 specific keywords used to find references
        -   saves the modified files in dataset.root_dir / 'fragments/' directory
    The extraction is done using multiprocessing. 
    :param redo         Extract the keywords from converted security policies from scratch.
    :param update_json  Same functionality as in 'get_certs_from_web()' method.
    """
    dataset.extract_keywords(redo, update_json)
    """
    This method is used to extract algorithm implementation IDs from tables in 
    security policy PDFs.
    The process may fail for some security policies, so a list of paths
    to not parsed files is returned.
    :param high_precision   This parameter specifies, whether to use a heuristic to find the pages with tables in the PDF, scanning only them. However, this is often not extensive and the results are not precise for some certificates.
                            If 'True', the scan is performed on all pages of every PDF.
                            If 'False', heuristics are used to find the pages containing tables.
    :param update_json      Same functionality as in 'get_certs_from_web()' method.
    :return                 List of paths of security policy PDFs that were not parsed.
    """
    not_decoded_files = dataset.extract_certs_from_tables(high_precision, update_json)
    """
    The algorithm information can be reused from previous runs and can be loaded in 
    the 'get_certs_from_web()' method. However, to always work with the most recent 
    information, it is recommended to always update the algorithm implementation dataset.
    """
    aset = FIPSAlgorithmDataset({}, Path(dataset.root_dir / 'web/algorithms'), 'algorithms', 'sample algs')
    aset.get_certs_from_web()
    dataset.algorithms = aset
    """
    This method is used to combine results from 'web_scan' and 'pdf_scan'.
    This method is redone everytime, even though the results may be saved.
    It implements the 'processed' functionality:
        -   sanitizes and unifies the extracted algorithm implementation and module IDs
        -   merges the results of 'web_scan' and 'pdf_scan'
        -   removes false positives by using heuristics
        -   finds the references for each certificate
    The search for not found algorithms from web in the PDF is also ran here.
    :param update_json      Same functionality as in 'get_certs_from_web()' method.
    """
    dataset.finalize_results(update_json)
    """
    This method is used to plot graphviz graphs of dependencies between the certificates.
    Three graphs are plotted - one only from 'web_scan', one only from 'pdf_scan'
    and one from processed results.
    The graphs are saved in 'dataset.root_dir / <graph_type>.png', where <graph_type>
    is one of ['full_graph', 'web_only_graph', 'pdf_only_graph'].
    :param show     Display the graphs right after plotting.
    """
    dataset.plot_graphs(show)
\end{lstlisting}

\section{Entrypoint}\label{entrypoint}

As we mentioned earlier, we provided \texttt{fips\_oop\_demo.py} as an entrypoint for the FIPS 140-2 functionality. The script provides CLI and logging for the functionality described in \myref{Listing}{code}.

The script accepts following arguments, all of which are optional:
\begin{itemize}
    \item \texttt{-{}-config-file} -- specifies the path to the YAML config file
    \item \texttt{-{}-json-file} -- specifies the path to the saved dataset from previous runs
    \item \texttt{-{}-no-download-algs} -- use the cached algorithm dataset, do not update it from web
    \item \texttt{-{}-redo-web-scan} -- flag, specifies whether to redo \texttt{web\_scan} information extraction from scratch
    \item \texttt{-{}-redo-keyword-scan} -- flag, specifies whether to redo \texttt{pdf\_scan} information extraction from scratch
    \item \texttt{-{}-higher-precision-results} -- flag, specifies whether to search all pages in PDFs for tables in order to obtain more precise results 
\end{itemize}
It is recommended to run the script without arguments the first time it is run.

The full extraction and analysis process presented in \texttt{fips\_oop\_demo.py} takes four to six hours and about four gigabytes of disk space is occupied by downloaded metadata used in computations.

More information on the \texttt{fips\_oop\_demo.py} entrypoint is available in the main README of \texttt{sec-certs}.

\renewcommand{\thechapter}{b}

\chapter{Data attachments}
\begin{itemize}
    \item \texttt{sec-certs} -- a snapshot of the \texttt{sec-certs} tool from the \texttt{sec-certs} github repository~\cite{sec-certs}.
    \item \texttt{graphs} -- folder containing the dot graphs of CMVP cryptographic module references plotted using the \texttt{sec-certs} tool.
\end{itemize}
  \backmatter
  \printindex
  \begingroup
  \sloppy
  \printbibliography
\endgroup


\end{document}
