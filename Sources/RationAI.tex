\chapter{Prostate Cancer Detection Setting}

According to the Masaryk Memorial Cancer Institute, prostate cancer is the most common oncological disease amongst men in the Czech Republic, with around $8$ thousand new cases reported each year.
In the global context, $1,276,106$ of new cases is estimated to appear solely in $2018$.
To aid pathologists in tackling the ever-growing number of new cases, the RationAI team trained and tested CNN on a dataset provided by Masaryk Memorial Cancer Institute.

This chapter briefly introduces prostate cancer, a contemporary approach to detecting malignant tissue, and finally, a dataset and RationAI's model for prostate cancer detection.

\section{Prostate Cancer}

%intro
Prostate cancer is a disease that causes rapid growth of cells in the prostate -- a male gland under the bladder.
This gland is responsible for producing seminal fluid that aids in the transport and nourishment of sperm.
The type of cancer attacking glandular tissue is called adenocarcinoma.

Prostate cancer is commonly diagnosed amongst men after $50$ years of age, and the incidence rate keeps increasing with aging -- we detect nearly $60$\% presence in men over $65$.
The mortality rate per $100,000$ people varies worldwide, ranging from $3.3$ in Eastern Asia to $10.7$ in Central America. Diet, physical activity, ethnicity, and family history are all likely to influence cancer's development and progression.
Implications of prostate cancer are notable even if it does not result in death, as it affects one's quality of life due to potential urinary, bowel, and sexual dysfunctions.
In advanced stages, prostate cancer can spread beyond the prostate gland, affecting the bladder, rectum, or bones []. % https://www.ncbi.nlm.nih.gov/pmc/articles/PMC6497009/pdf/wjon-10-063.pdf
Given the high prevalence and potential for aggressiveness, early detection and effective management of prostate cancer are vital to improving outcomes and survival rates.

\section{Gleason Patterns}

To effectively detect and estimate the impact of adenocarcinoma spread in the prostate, in 1996, Donald Gleason presented a unified grading and scoring system. Gleason's system gained universal acceptance in 1974 and is the most prevalent system doctors use today. It categorizes the growth of cancer cells into distinctive patterns based on how much the cancerous tissue differs from healthy prostate gland cells, as shown in Figure \ref{fig:gp}. Several patterns and their descriptions have been refined later; as of today, we differ between 9 patterns. Complete enumeration and description of patterns can be found in [].
% https://onlinelibrary.wiley.com/doi/epdf/10.1111/j.1365-2559.2011.04003.x?saml_referrer

To calculate a Gleason score, a histopathologist determines the predominant and second most common Gleason pattern -- the final score is simply a sum of pattern category numbers. The influence of score on a doctor's evaluation has also evolved. The International Society of Urological Pathology (ISUP) distinguishes between $5$ grades of prostate cancer. More on scoring and grading innuendos can be found in []. In 2016, WHO refined the grading, which is standard up to this day.
\todo{Make sense of this.}
% file:///Users/krebso/kiwi/outpayments/apps/trxgate/CANCER%20DE%20PROSTATA%20-%20REVISAO%20DE%20DIAG%20E%20TRATAMENTO.pdf

\begin{figure}
    \begin{center}
    \begin{minipage}{1\textwidth}
      \includegraphics[width=\textwidth]{img/gp-classification.png}
    \end{minipage}
    \caption{Types of Gleason Pattern ranked from $1$ to $5$. Taken from []}
    \label{fig:gp}
    \end{center}
% https://www.mdpi.com/2072-6694/14/23/5897
\end{figure}

\section{Dataset}

Masaryk Memorial Cancer Institute provided RationAI with pseudonymized dataset of hematoxylin/eosin stained WSIs.
Each WSI contains from 3-5 biopsies and is stored as a MIRAX uncompressed PNG of size $105,185 px \times 221,772 px$.
The dataset is stratified into training and test split, containing $264$ WSIs with cancer, $436$ without and $37$ WSIs with cancer and $50$ without respectively.
WHO grade distribution can be found in Table \ref{tab:who_grade_distribution}.

\begin{table}
\centering
\ra{1.2}
\begin{tabular}{@{}lll@{}}\toprule
WHO Grade Distribution & Training split & Test split \\ 
\midrule
Grade 1         & 38 cases            & 5 cases      \\
Grade 2         & 31 cases            & 1 case       \\
Grade 3         & 16 cases            & 1 case       \\
Grade 4         & 9 cases             & 1 case       \\
Grade 5         & 10 cases            & 2 case       \\
\bottomrule
\end{tabular}
\caption{WHO grade distributions for Training and Test split of prostate WSI dataset.}
\label{tab:who_grade_distribution}
\end{table}

To mark cancerous areas, WSIs containing adenocarcinomas were manually annotated by domain expert -- annotations are in form of polygons, encapsulating cancerous areas.
Because processing of whole WSI at once is computationaly expensive, we used common technique of tiling the WSIs to decrease memory requirements.
Each WSI is split into overlapping tiles of size $512px \times 512px$.
An overlap of $256$ pixels is used between two consecutive tiles, to counter possibility of severing important patterns.
This is a well established approach used industry-wide, as per [].

WSIs contain several biopsies with blank space in between them.
To chisel out only relevant tiles, any tile with less than $50$\% of its area covered by tissue is discarded from both training and testing dataset splits.
Consequently, while all WSIs have the same size, their usable tile count differs.

\section{Model}

Model trained by RationAI to perform cancer detection in WSIs is a modified version of popular CNN architecture VGG16 [].
The full architecture breakdown is portrayed in Figure \ref{fig:rationai-vgg16}.
% https://arxiv.org/abs/1409.1556

The model aims to solve following problem, loosely paraphrasing Gallo et al. in [gallo]: Classify tiles according to the presence/absence of cancer in their central square area of size $256 px \times 256 px$. Presence is determined by whether the central squares overlaps with annotation from domain expert. Model achieves very good score on the test split, reaching $100$\% slide-level prediction accuracy.

\todo{Ask Vit about where to scavenge params, maybe mention camelyon?}

\begin{figure}
    \begin{center}
    \begin{minipage}{0.75\textwidth}
      \includegraphics[width=\textwidth]{img/nn-arch.png}
    \end{minipage}
    \caption{Architecture of Prostate Cancer Binary Classifier model utilized by RationAI. Model is based on VGG-16, introduced by Simonyan and Zisserman in 2014 []. Original VGG-16 model is trained to classify $1000$ classes from ILSVRC dataset [] and utilizes three FC layers before applying softmax on the last one. In our solution, global max pooling is placed before single fully connected layer, reducing each of the $512$ activation maps into single value. This can be seen as an extension of logistic regression, with condensed activation maps as independent variables. Sigmoid is applied on output of the FC layer, to turn it into probabilistic distribution [].}
    \label{fig:rationai-vgg16}
    \end{center}
%% original vgg paper
% ILSVRC dataset
%% how sigmoid makes it probability or whatever
\end{figure}

