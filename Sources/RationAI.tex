\chapter{Prostate Cancer Detection Setting}

According to the Masaryk Memorial Cancer Institute, prostate cancer is the most common oncological disease amongst men in the Czech Republic, with around $8$ thousand new cases reported each year \cite{mmci-prostate-cancer}.
In the global context, experts estimate $1,276,106$ of new cases appeared solely in $2018$ \cite{world-prostate-cancer}.
To aid pathologists in tackling the ever-growing number of new cases, the RationAI team trained and tested a CNN on a dataset provided by Masaryk Memorial Cancer Institute.

This chapter briefly introduces prostate cancer, a contemporary approach to detecting malignant tissue, and finally, a dataset and RationAI's model for prostate cancer detection.

\section{Prostate Cancer}

Prostate cancer is a disease that causes rapid growth of cells in the prostate -- a male gland under the bladder.
This gland is responsible for producing seminal fluid that aids in the transport and nourishment of sperm.
The type of cancer attacking glandular tissue is called adenocarcinoma.

Doctors commonly diagnose prostate cancer amongst men after $50$ years of age, and the incidence rate keeps increasing with aging --- we detect nearly $60$\% presence in men over $65$.
The mortality rate per $100,000$ people varies worldwide, ranging from $3.3$ in Eastern Asia to $10.7$ in Central America. Diet, physical activity, ethnicity, and family history are all likely to influence cancer's development and progression.
Implications of prostate cancer are notable even if it does not result in death, as it affects one's quality of life due to potential urinary, bowel, and sexual dysfunctions.
In advanced stages, prostate cancer can spread beyond the prostate gland, affecting the bladder, rectum, or bones \cite{world-prostate-cancer}.
Given the high prevalence and potential for aggressiveness, early detection and effective management of prostate cancer are vital to improving outcomes and survival rates.

\subsection*{Gleason Patterns and Score}

To effectively detect and estimate the impact of adenocarcinoma spread in the prostate, in 1996, Donald Gleason presented a unified grading and scoring system.
Gleason's system gained acceptance in 1974 and is the most prevalent system doctors use today.
It categorizes the growth of cancer cells into distinctive patterns based on how much the cancerous tissue differs from healthy prostate gland cells, as shown in \myref{Figure}{fig:gp}.
Several patterns and their descriptions have been refined later; today, we differ between 9 patterns. Complete enumeration and description of patterns can be found in \cite{gleason-patterns}.

To calculate a Gleason score, the histopathologist determines the predominant and second most common Gleason pattern -- the final score is simply a sum of pattern category numbers. The International Society of Urological Pathology distinguishes between $5$ grades of prostate cancer. More on scoring and grading innuendos can be found in \cite{gleason-pattern-grading}.
In 2016, WHO refined the grading into so-called \emph{grade groups} \cite{who-grade-groups}.

\begin{figure}
    \begin{center}
    \begin{minipage}{1\textwidth}
      \frame{\includegraphics[width=\textwidth]{img/gp-classification.png}}
    \end{minipage}
    \caption{Example of Gleason Patterns ranked from $1$ to $5$. Pattern $1$ represents a healthy stroma --- ``cells and tissues that support and give structure to organs, glands, or other tissues in the body'', per definition from \cite{nci-stroma}. Pattern $5$ represents tissue with the highest risk of cancer. Sourced from \cite{gleason-pattern-description}}
    \label{fig:gp}
    \end{center}
\end{figure}

\section{Dataset}\label{sec:dataset}

Masaryk Memorial Cancer Institute provided the RationAI group with a pseudonymized hematoxylin/eosin-stained WSIs dataset.
Each WSI contains $3$ to $5$ biopsies and is stored as a MIRAX uncompressed PNG of size $105,185 \times 221,772$ pixels.
The dataset is stratified into training and test split, containing $264$ WSIs with cancer, $436$ without, $37$ WSIs with cancer, and $50$ without, respectively \cite{gallo}.
\myref{Table}{tab:who_grade_distribution} shows the WHO grade group distribution.

\begin{table}
\centering
\ra{1.2}
\begin{tabular}{@{} l l l @{}}\toprule
WHO Grade Distribution & Training split & Test split \\ 
\midrule
Grade 1         & 38 cases            & 5 cases      \\
Grade 2         & 31 cases            & 1 case       \\
Grade 3         & 16 cases            & 1 case       \\
Grade 4         & 9 cases             & 1 case       \\
Grade 5         & 10 cases            & 2 case       \\
\bottomrule
\end{tabular}
\caption{WHO grade distributions for Training and Test split of prostate WSI dataset.}
\label{tab:who_grade_distribution}
\end{table}

\todo{in matejs paper, there is something about the distribution negatively affecting some results.}
To mark cancerous areas, domain experts manually annotated areas of WSIs containing adenocarcinomas -- annotations are polygons encapsulating cancerous regions.
Because processing the whole WSI at once is computationally expensive, we tile the WSIs to decrease memory requirements.
Each WSI is split into overlapping tiles of size $512 \times 512$ pixels.
An overlap of $256$ pixels is used between two consecutive tiles to counter the possibility of severing critical patterns.
Such tiling is a well-established approach used industry-wide, as per [].

WSIs contain several biopsies with blank spaces in between them.
To chisel out only relevant parts of WSI, any tile with less than $50$\% of its area covered by tissue is discarded from training and testing dataset splits.
Consequently, while all WSIs have the same size, their usable tile count differs.

\subsection*{xPOI Dataset}\label{xpoi-dataset}

The WSI dataset is vast, and to verify that models trained by the RationAI group focus on the important morphological features, \cite{gallo} sampled a small subset of this dataset to present to a domain expert upon model's evaluation.
This dataset consists of $461$ tiles.\todo{check actual count and write something in there}.

\section{Model}\label{model}

A model trained by RationAI to perform cancer detection in tiled WSIs is a modified version of the well-known CNN architecture VGG16, introduced in \cite{vgg16}.
\myref{Figure}{fig:rationai-vgg16} portrays the full model breakdown.

Loosely paraphrasing \cite{gallo}, the model aims to solve the following problem: \emph{Classify tiles according to the presence/absence of cancer in their central square area of $256 \times 256 $ pixels. Presence is determined by whether the central square overlaps with the annotation from the domain expert}. The model achieves a notably good score on the test split, reaching $100$\% slide-level prediction accuracy --- meaning the model correctly determines whether a WSI contains cancer for each of the test WSIs.

However, blindly trusting the model in such a high-stake environment, as cancer detection undoubtedly is, based on accuracy, is not enough.
To verify that the model bases its decisions on relevant features, \Citeauthor*{gallo} Gallo et at.\todo{fix smaller font} conducted an additional experiment, presenting annotated xPOI Dataset from \myref{Section}{xpoi-dataset} to a domain expert.
Using the technique described in the \myref{Subsection}{occlusion}, Gallo et al. annotated those data to display models' areas of interest.
The domain expert concluded that the model focuses on critical morphological features \cite{gallo}.

\begin{figure}
    \begin{center}
    \begin{minipage}{0.75\textwidth}
      \includegraphics[width=\textwidth]{img/nn-arch.png}
    \end{minipage}
    \caption{Architecture of Prostate Cancer Binary Classifier model utilized by the RationAI group. The model is based on VGG16, introduced by Simonyan and Zisserman in \cite{vgg16}. They originally trained their VGG16 model to classify $1000$ classes from the ILSVRC dataset \cite{ilsvrc} and utilized three FC layers before applying softmax on the last one. RationAIs' solution places global max pooling before a single FC layer, reducing each of the $512$ pooled activation maps to a single value. Sigmoid is applied to the output of the FC layer to squish the raw value in the range $(0, 1)$. This can be seen as an extension of the logit model, with condensed activation maps as independent variables. Therefore, we can interpret the final output as a probability of pro-cancer markers in a given tile's central $256 \times 256$ pixel square.}
    \label{fig:rationai-vgg16}
    \end{center}
\end{figure}

